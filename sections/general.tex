\chapter{Allgemeines}
  \begin{example}[Ableitungen und Stammfunktionen]
  	\renewcommand{\arraystretch}{1.5}
    \begin{tabular}{| c | c | c |}
          \hline 
      $f(x) = $         & $f'(x) = $                            & $F(x) = $\\ %header
          \hhline{|=|=|=|}
      $c$               & $0$                                   & $c\cdot x$\\
          \hline
      $x^n$             & $n\cdot x^{n-1}$                      & $\frac{1}{n+1} \cdot x^{n+1}$\\
          \hline
      $\frac{1}{x}$     & $-\frac{1}{x^2}$                      & $\ln{|x|}$\\
          \hline
      $\ln{|x|}$        & $\frac{1}{x}$                         & $x \cdot (\ln{|x|}-1)$\\
          \hline
      $\log_a{|x|}$     & $\frac{1}{x \cdot \ln{|a|}}$          & $x \cdot (\log_a{|x|}-\frac{1}{ln{|a|}})$\\
          \hline
      $a^x$             & $\ln a\cdot a^x$                      & $a^x \cdot \frac{1}{\ln|a|}$\\
          \hline
      $\sin x$          & $\cos x$                              & $-\cos x$\\
          \hline
      $\cos x$          & $-\sin x$                             & $\sin x$\\
          \hline
      $\tan x$          & $1 + \tan^2 x = \frac{1}{\cos^2 x}$   & $-\ln|\cos x|$\\
          \hline
      $\sin^2 x$        & $2\sin x\cos x$                       & $\frac{1}{2}(x-\sin x\cos x)$\\
          \hline
      $\cos^2 x$        & $-2\sin x\cos x$                      & $\frac{1}{2}(x+\sin x\cos x)$\\
          \hline
      $\tan^2 x$        & $2 \cdot \frac{tan x}{\cos^2 x}$      & $t\tan x - x$\\
          \hline
      $\arcsin x$       & $\frac{1}{\sqrt{1-x^2}}$              & $x \cdot \arcsin{x} + \sqrt{1-x^2}$\\
          \hline
      $\arccos x$       & $-\frac{1}{\sqrt{1-x^2}}$             & $x \cdot \arccos{x} - \sqrt{1-x^2}$\\
          \hline
      $\arctan x$       & $\frac{1}{1+x^2}$                     & $x \cdot \arctan{x} + \frac{1}{2}\ln{(1+x^2)}$\\
          \hline
      $\sinh x$         & $\cosh x$                             & $\cosh x$\\
          \hline
      $\cosh x$         & $\sinh x$                             & $\sinh x$\\
          \hline
      $\tanh x$         & $1 - \tanh^2 x = \frac{1}{\cosh^2 x}$ & $\ln{(\cosh x)}$\\
          \hline
    \end{tabular}
  \end{example}
  
  \begin{example}[Wichtige Reihen \& Limits] $ $\\
    \begin{itemize}
      \item geometrische Reihe: $\sum_{k=0}^{\infty} q^k$ = $\dfrac{1}{1-q}$ ist konvergent für $|q| < 1$, da $\sum_{k=0}^{n} q^k$ = $\dfrac{1-q^{n+1}}{1-q}$
      \item harmonische Reihe: $\sum_{k=1}^{\infty} \dfrac{1}{k}$ ist divergent
      \item alternierende harmonische Reihe: $\sum_{k=1}^{\infty} \dfrac{(-1)^{k-1}}{k}$ ist konvergent, aber nicht absolut konvergent
      \item Leibnizreihen haben die Form $\sum_{k=1}^{\infty} (-1)^{k-1}a_k$ und sind konvergent
      \item $\sum_{k=0}^{\infty} \dfrac{1}{k^j}$ ist konvergent für $j \geq 2$, keine Aussagen über $1<j<2$
      \item Euler-Mascheroni Konstante $\lim_{n \to \infty} \big(\sum_{k=1}^n \frac{1}{k} - \ln n \big)$
    \end{itemize}
  \end{example}
  
  \begin{example}[Infinite Series]
  $ $\\
  	\renewcommand{\arraystretch}{1.5}
  	\begin{tabular}{| l | l |}
  			\hline
    	$\sum_{n=0}^\infty (k+1) \cdot q^n + \frac{1}{(1-q)^2}, \; |q| < 1$ & $\sum_{n=0}^\infty a \cdot q^n + \frac{a}{1-q}, \; |q| < 1$ \\
        	\hline
        $\sum_{n=0}^\infty \frac{(-1)^k}{2k+1} = \pi /4$ & $\sum_{n=1}^\infty \frac{(-1)^{k+1}}{k} = \ln 2$ \\
        	\hline
        $\sum_{n=1}^\infty \frac{1}{k^2} = \pi^2 / 6$ & $\sum_{n=1}^\infty \frac{(-1)^{k+1}}{k} = \pi^2 / 12$ \\
        	\hline
  	\end{tabular}
  \end{example}
  
  \begin{example}[Other Important Stuff] $ $
  	\begin{itemize}
        \item $\sum_{k=1}^{n}k = \frac{n(n+1)}{2}$
		\item $\sum_{k=1}^{n}k^2 = \frac{n(n+1)(2n+1)}{6}$
		\item $\sum_{k=1}^{n}k^3 = (\frac{n(n+1)}{2})^2= (\sum_{k=1}^{n}k)^2$
		\item $\sum_{k=1}^{n}(2k-1) = n^2$
		\item $\sum_{k=1}^{n}(2k-1)^2 = \frac{n(2n-1)(2n+1)}{3}$
        \item $ \sin^2 x + cos^2x = 1 $
		\item $ \sin (x+y) = \sin x \cdot \cos y + \cos x \cdot \sin y $
		\item $ \cos (x+y) = \cos x \cdot \cos y - \sin x \cdot \sin y $
		\item $ \sin(2\alpha) = 2 \sin(\alpha)\cos(\alpha)$
        \item $ \sin x = \frac{\exp{(ix)} - \exp {(-ix)}}{2i} $
        \item $ \cos x = \frac{\exp{(ix)} + \exp{(-ix)}}{2} $
        \item $ \sinh{x} = \frac{e^x - e^{-x}}{2} $
        \item $ \cosh{x} = \frac{e^x + e^{-x}}{2} $
        \item $ \sin^2{x} = \frac{1 - \cos{(2x)}}{2} $
        \item $ \cos^2{x} = \frac{1 + \cos{(2x)}}{2} $
        \item $ \tan^2{x} = \frac{1 - \cos{(2x)}}{1 + \cos{(2x)}} $
        \item $ \sin(\arccos x) = \sqrt{1-x^2} $ \tab $ \sin(\arctan x) = x/\sqrt{1+x^2} $
        \item $ \cos(\arcsin x) = \sqrt{1-x^2} $ \tab $ \cos(\arctan x) = 1 / \sqrt{1+x^2} $
        \item $ \tan(\arcsin x) = x/\sqrt{1-x^2} $ \tab $ \tan(\arccos x) = \sqrt{1-x^2}/x $
  	\end{itemize}
  \end{example}
  
  \begin{example} $ $\\
  	\renewcommand{\arraystretch}{1.2}
  	\begin{tabular}{| c | c || c | c | c |}
  			\hline
        Radian & Gradian & $\sin$ & $\cos$ & $\tan$ \\
        	\hline
        $0\deg$ & $0$ & $0$ & $1$ & $0$ \\
        	\hline
        $30\deg$ & $\pi/6$ & $1/2$ & $\sqrt{3}/2$ & $\sqrt{3}/3$ \\
        	\hline
        $45\deg$ & $\pi/4$ & $\sqrt{2}/2$ & $\sqrt{2}/2$ & $1$ \\
        	\hline
        $60\deg$ & $\pi/3$ & $\sqrt{3}/2$ & $1/2$ & $\sqrt{3}$ \\
        	\hline
        $90\deg$ & $\pi/2$ & $1$ & $0$ & $\infty$ \\
        	\hline
        $120\deg$ & $2\pi/3$ & $\sqrt{3}/2$ & $-1/2$ & $-\sqrt{3}$ \\
        	\hline
        $135\deg$ & $3\pi/4$ & $\sqrt{2}/2$ & $-\sqrt{2}/2$ & $-1$ \\
        	\hline
        $150\deg$ & $5\pi/6$ & $1/2$ & $-\sqrt{3}/2$ & $-\sqrt{3}/3$ \\
        	\hline
        $180\deg$ & $\pi$ & $0$ & $-1$ & $0$ \\
        	\hline
  	\end{tabular}
  \end{example}

    \section{Betrag}
	
	$|ab|=|a||b|$\\
    $|\frac{a}{b}|=\frac{|a|}{|b|}$\\
	$|a+b|\leq |a|+|b|$
	
	
	\section{Potenzen und Wurzeln}
    
	\subsection{Definition} 
	
	$x= \sqrt[n]{a} \Leftrightarrow (x^n = a \ und \ x\geq 0)$\\
	Es folgt:$\sqrt[n]{-a}=-\sqrt[n]{a}, a \geq 0$\\
	$a^{-n} = \frac{1}{a^n}= (\frac{1}{a})^n$\\
	$a^{\frac{1}{a}}= \sqrt[n]{a}$ \tab $\sqrt{ab}=\sqrt{a}\sqrt{b}$\\
	$a^{\frac{m}{n}}= \sqrt[n]{a^m}$\tab $\sqrt{\frac{a}{b}}= \frac{\sqrt{a}}{\sqrt{b}}$\\
	$a^x= e^{x\cdot \ln a}$ \tab $\sqrt[n]{a^{-m}}= \frac{1}{\sqrt[n]{a^m}}$
	
    \subsection{Potenzgesetzte und Wurzelgesetze}
    
	$a^m a^n = a^{m+n}$ \tab $\sqrt[n]{a^m} = \sqrt[kn]{a^{km}}$
	
	\noindent$\frac{a^m}{ a^n} = a^{m-n}$ \tab $\sqrt[n]{\sqrt[k]{a}} = \sqrt[nk]{a^{km}}$
	
	\noindent$(a^m)^n= a^{mn}$\tab $\sqrt[n]{a}\sqrt[n]{b}=\sqrt[n]{ab}$
	
	\noindent$a^n b^n = (ab)^n$ \tab $\frac{\sqrt[n]{a}}{\sqrt[n]{b}}= \sqrt[n]{\frac{a}{b}}$
	
	\noindent$\frac{a^n}{b^n} = (\frac{a}{b})^n$\\


	
    \section{Logarithmensätze}
    
	$\log(uv) = \log u +\log v$ \hspace{0.05cm} $\log(\frac{u}{v}) = \log u -\log v$
	
	\noindent$\log(u^r) = r\cdot \log u$ \tab $\log(\frac{1}{v}) = -\log v$\\
	
	\section{Determinante}
	
	$A\in\mathbb{R}^2$: $|A|=ad-bc$\\
	$A\in\mathbb{R}^3$: $|A|=aei+bfg+cdh-ceg-bdi-afh$
	
	\section{Inverse}
	
	$$A = \begin{pmatrix} 
        a   & b \\
        c   & d
    \end{pmatrix} 
    \Longrightarrow 
    A^{-1} = \frac{1}{\det A} \begin{pmatrix} 
        d   & -b \\
        -c   & a
    \end{pmatrix}$$
    
    \section{Ableitungsregeln}

    \begin{itemize}
        \item $(af + bg)' = af' + bg'$ 
        \item $(f \cdot g)' = f'\cdot g + f \cdot g'$
        \item $(\frac{f}{g})' \frac{f'\cdot g - f \cdot g'}{g^2}$
        \item $f(g(x))' = f'(g) \cdot g'$
    \end{itemize}
    
    \section{Integrationsregeln in $\R$}
    
    \subsection{Partielle Integration}
    
    $\int u \cdot v' dx = uv - \int u' \cdot v dx$ \\
    $v': x^n, \frac{1}{1-x^2}, \frac{1}{1+x^2}, 1, \dots$ \\
    $u : x^n, \log x, arc-Funktionen, \dots$ \\
    egal: $e^x, \sin x, \cos x, \dots$ \\
    Bemerkung: mit 1 multiplizieren wenn $\log$ oder $\exp$ vorkommen \\
    Bsp: $xe^x dx = xe^x - \int 1 \cdot e^x dx = xe^x - e^x + C$
    
    \subsection{Direkte Integrale}
    
    $\int f(g(x)) \cdot g'(x) dx = F(g(x)) + C$ \\
    Bsp: $\int \frac{1}{x\cdot \ln x} dx = \log\log x + C$
    
    \subsection{Rationale Funktionen}
    
    $\int \frac{p(x)}{q(x)}$, wobei $p(x), q(x)$ Polynome sind \\
    \begin{itemize}
        \item falls $deg(p) \geq deg(q) \Longrightarrow Polynomdivision$ \\
        \item falls $deg(p) < deg(q) \Longrightarrow PBZ$
    \end{itemize}
    
    \subsection{Substitution}
    
    $\int_{g(a)}^{g(b)} f(x) dx = \int_a^b f(g(t)) \cdot g'(t) dt$ \\
    Bsp: $\int \frac{e^x}{e^x+1} dx, \: e^x = t \Longrightarrow x = \log t, dx = \frac{1}{t} dt$ \\
    $\Longrightarrow \int \frac{t^2}{t + 1} \cdot \frac{1}{t} dt = \dots = e^x - \log(e^x+1)+C$ \\

\vfill
