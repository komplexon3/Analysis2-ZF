 %% (Master) Thesis template
% Template version used: v1.4
%
% Largely adapted from Adrian Nievergelt's template for the ADPS
% (lecture notes) project.


%% We use the memoir class because it offers a many easy to use features.
\documentclass[9pt,a4paper, oneside, twocolumn]{memoir}
%\documentclass[11pt,a4paper,titlepage]{memoir}

%% Packages
%% ========

%% LaTeX Font encoding -- DO NOT CHANGE
\usepackage[OT1]{fontenc}

%% Babel provides support for langua      .//ges.  'english' uses Britishs
%% English hyphenation and text snippets like "Figure" and
%% "Theorem". Use the option 'ngerman' if your document is in German.
%% Use 'american' for American English.  Note that if you change this,
%% the next LaTeX run may show spurious errors.  Simply run it again.
%% If they persist, remove the .aux file and try again.
\usepackage[english]{babel}

%% Input encoding 'utf8'. In some cases you might need 'utf8x' for
%% extra symbols. Not all editors, especially on Windows, are UTF-8
%% capable, so you may want to use 'latin1' instead.
\usepackage[utf8]{inputenc}

%% This changes default fonts for both text and math mode to use Herman Zapfs
%% excellent Palatino font.  Do not change this.
\usepackage[sc]{mathpazo}

%% The AMS-LaTeX extensions for mathematical typesetting.  Do not
%% remove.
\usepackage{amsmath,amssymb,amsfonts,mathrsfs}

%% NTheorem is a reimplementation of the AMS Theorem package. This
%% will allow us to typeset theorems like examples, proofs and
%% similar.  Do not remove.
%% NOTE: Must be loaded AFTER amsmath, or the \qed placement will
%% break
\usepackage[amsmath,thmmarks]{ntheorem}

%% LaTeX' own graphics handling
\usepackage{graphicx}

%% We unfortunately need this for the Rules chapter.  Remove it
%% afterwards; or at least NEVER use its underlining features.
\usepackage{soul}

%% This allows you to add .pdf files. It is used to add the
%% declaration of originality.
\usepackage{pdfpages}

%% Some more packages that you may want to use.  Have a look at the
%% file, and consult the package docs for each.
%% See the TeXed file for more explanations

%% [OPT] Multi-rowed cells in tabulars
%\usepackage{multirow}

%% [REC] Intelligent cross reference package. This allows for nice
%% combined references that include the reference and a hint to where
%% to look for it.
\usepackage{varioref}

%% [OPT] Easily changeable quotes with \enquote{Text}
%\usepackage[german=swiss]{csquotes}

%% [REC] Format dates and time depending on locale
\usepackage{datetime}

%% [OPT] Provides a \cancel{} command to stroke through mathematics.
%\usepackage{cancel}

%% [NEED] This allows for additional typesetting tools in mathmode.
%% See its excellent documentation.
\usepackage{mathtools}

%% [ADV] Conditional commands
%\usepackage{ifthen}

%% [OPT] Manual large braces or other delimiters.
%\usepackage{bigdelim, bigstrut}

%% [REC] Alternate vector arrows. Use the command \vv{} to get scaled
%% vector arrows.
\usepackage[h]{esvect}

%% [NEED] Some extensions to tabulars and array environments.
\usepackage{array}

%% [OPT] Postscript support via pstricks graphics package. Very
%% diverse applications.
%\usepackage{pstricks,pst-all}

%% [?] This seems to allow us to define some additional counters.
%\usepackage{etex}

%% [ADV] XY-Pic to typeset some matrix-style graphics
%\usepackage[all]{xy}

%% [OPT] This is needed to generate an index at the end of the
%% document.
%\usepackage{makeidx}

%% [OPT] Fancy package for source code listings.  The template text
%% needs it for some LaTeX snippets; remove/adapt the \lstset when you
%% remove the template content.
\usepackage{listings}
\lstset{language=TeX,basicstyle={\normalfont\ttfamily}}

%% [REC] Fancy character protrusion.  Must be loaded after all fonts.
\usepackage[activate]{pdfcprot}

%% [REC] Nicer tables.  Read the excellent documentation.
\usepackage{booktabs}

%% Package for writing algorithms
\usepackage{algorithmicx,algpseudocode}

%% Package for continuing enumerations
\usepackage{enumitem}

% my own shit
\usepackage{multirow}
\usepackage{multicol}
\usepackage{adjustbox}

\usepackage{mdframed}
\usepackage{ntheorem}

\usepackage{hhline}

%% Our layout configuration.  DO NOT CHANGE.
%% Memoir layout setup

%% NOTE: You are strongly advised not to change any of them unless you
%% know what you are doing.  These settings strongly interact in the
%% final look of the document.

% Turn extra space before chapter headings off.
\setlength{\beforechapskip}{0pt}

\nonzeroparskip
\parindent=0pt
\defaultlists

% Chapter style redefinition
\makeatletter

%\if@twoside
%  \pagestyle{Ruled}
%  \copypagestyle{chapter}{Ruled}
%\else
%  \pagestyle{ruled}
%  \copypagestyle{chapter}{ruled}
%\fi
%\makeoddhead{chapter}{}{}{}
%\makeevenhead{chapter}{}{}{}
%\makeheadrule{chapter}{\textwidth}{0pt}
%\copypagestyle{abstract}{empty}

%\makechapterstyle{bianchimod}{%
%  \chapterstyle{default}
%  \renewcommand*{\chapnamefont}{\normalfont\Large\sffamily}
%  \renewcommand*{\chapnumfont}{\normalfont\Large\sffamily}
%  \renewcommand*{\printchaptername}{%
%    \chapnamefont\centering\@chapapp}
%  \renewcommand*{\printchapternum}{\chapnumfont {\thechapter}}
%  \renewcommand*{\chaptitlefont}{\normalfont\huge\sffamily}
%  \renewcommand*{\printchaptertitle}[1]{%
%    \hrule\vskip\onelineskip \centering \chaptitlefont\textbf{\vphantom{gyM}##1}\par}
%  \renewcommand*{\afterchaptertitle}{\vskip\onelineskip \hrule\vskip
%    \afterchapskip}
%  \renewcommand*{\printchapternonum}{%
%    \vphantom{\chapnumfont {9}}\afterchapternum}}

% Use the newly defined style
%\chapterstyle{bianchimod}

%\makechapterstyle{bianchimod}{%
%  \chapterstyle{default}
%  \renewcommand*{\chapnamefont}{\normalfont\Large\sffamily}
%  \renewcommand*{\chapnumfont}{\normalfont\Large\sffamily}
%%  \renewcommand*{\printchaptername}{}
%%  \renewcommand*{\printchapternum}{}
%  \renewcommand*{\chaptitlefont}{\normalfont\huge\sffamily}
%  \renewcommand*{\printchaptertitle}[1]{%
%    \chaptitlefont\textbf{\thechapter\ \vphantom{gyM}##1}\par}}
%%  \renewcommand*{\afterchaptertitle}{\afterchapskip}
%  \renewcommand*{\printchapternonum}{\vphantom{\chapnumfont {1}}\afterchapternum}
%}

% Use the newly defined style
%\chapterstyle{bianchimod}




% change section style
\makeatletter
\renewcommand{\chapter}{\@startsection{chapter}{1}{0mm}%
                                {-1ex plus -0.0ex minus -0.0ex}%
                                {1ex plus 0.1ex}%x
                                {\normalfont\Large\bfseries}}
\renewcommand{\section}{\@startsection{section}{1}{0mm}%
                                {1ex plus -0.0ex minus -0.0ex}%
                                {2ex plus 0.0ex}%x
                                {\normalfont\normalsize\bfseries}}
\renewcommand{\subsection}{\@startsection{subsection}{2}{0mm}%
                                {-1ex plus -.5ex minus -.2ex}%
                                {0.5ex plus .2ex}%
                                {\normalfont\small\bfseries}}
\renewcommand{\subsubsection}{\@startsection{subsubsection}{3}{0mm}%
                                {-1ex plus -.5ex minus -.2ex}%
                                {1ex plus .2ex}%
                                {\normalfont\small\bfseries}}
\makeatother


\setsecheadstyle{\Large\bfseries\sffamily}
\setsubsecheadstyle{\large\bfseries\sffamily}
\setsubsubsecheadstyle{\bfseries\sffamily}
\setparaheadstyle{\normalsize\bfseries\sffamily}
\setsubparaheadstyle{\normalsize\itshape\sffamily}
\setsubparaindent{0pt}

% Set captions to a more separated style for clearness
\captionnamefont{\sffamily\bfseries\footnotesize}
\captiontitlefont{\sffamily\footnotesize}
\setlength{\intextsep}{16pt}
\setlength{\belowcaptionskip}{1pt}

% Set section and TOC numbering depth to subsection
\setsecnumdepth{subsection}
\settocdepth{subsection}

%% Titlepage adjustments
\pretitle{\vspace{0pt plus 0.7fill}\begin{center}\HUGE\sffamily\bfseries}
\posttitle{\end{center}\par}
\preauthor{\par\begin{center}\let\and\\\Large\sffamily}
\postauthor{\end{center}}
\predate{\par\begin{center}\Large\sffamily}
\postdate{\end{center}}

\def\@advisors{}
\newcommand{\advisors}[1]{\def\@advisors{#1}}
\def\@department{}
\newcommand{\department}[1]{\def\@department{#1}}
\def\@thesistype{}
\newcommand{\thesistype}[1]{\def\@thesistype{#1}}

%\renewcommand{\maketitlehooka}{\noindent\ETHlogo[2in]}

\renewcommand{\maketitlehookb}{\vspace{1in}%
  \par\begin{center}\Large\sffamily\@thesistype\end{center}}

\renewcommand{\maketitlehookd}{%
  \vfill\par
  \begin{flushright}
    \sffamily
%    \@advisors\par
%    \@department ETH Z\"urich
  \end{flushright}
}

\checkandfixthelayout

\setlength{\droptitle}{-48pt}

\makeatother

% This defines how theorems should look. Best leave as is.
\theoremstyle{plain}
\setlength\theorempostskipamount{0pt}

\pagestyle{empty}

\usepackage{vmargin}		%for margin settings
\RequirePackage[utf8]{inputenc}	
\setmarginsrb  {0.10in}  % left margin
               {0.30in}  % top margin
               {0.10in}  % right margin
               {0.10in}  % bottom margin
               {   0pt}  % head height
               {0.00in}  % head sep
               {   3pt}  % foot height
               {0.20in}  % foot sep

%%% Local Variables:
%%% mode: latex
%%% TeX-master: "thesis"
%%% End:


%% Theorem environments.  You will have to adapt this for a German
%% thesis.
%% Theorem-like environments

%% This can be changed according to language. You can comment out the ones you
%% don't need.

\numberwithin{equation}{chapter}

%\theorembodyfont{\normalfont}

%% English variants
\newtheorem{theoremm}{T}[chapter]
\newtheorem{examp}[theoremm]{E}
\newtheorem{rem}[theoremm]{R}
\newtheorem{coro}[theoremm]{C}
\newtheorem{defi}[theoremm]{D}
\newtheorem{lemma}[theoremm]{L}
\newtheorem{proposition}[theoremm]{P}
\newtheorem{conc}[theoremm]{Con}
\newcommand{\thistheoremname}{}
\newtheorem*{genericthm}{\thistheoremname}
\newenvironment{namedtheorem}[1]
  {\renewcommand{\thistheoremname}{#1}%
   \begin{genericthm}}
  {\end{genericthm}}
  
  
%\newenvironment{concept}[1]
%{
  % This is what comes before the content
%  {\textbf{#1\,}}
%}
%{
  % This is what comes after the content
%}

% \makeatletter
% \def\thm@space@setup{%
%   \thm@preskip=100pt \thm@postskip=100pt
% }
% \makeatother

 %\renewtheoremstyle{customstyle} %
 %{3pt} % Space above
 %{3pt} % Space below
 %{} % Body font
 %{} % Indent amount
 %{\upshape} % Theorem head font
 %{} % Punctuation after theorem head
 %{} % Space after theorem head
 %{} % Theorem head spec (can be left empty, meaning `normal')


%\theoremstyle{definition} 
% \newtheorem{examp}{Theorem}{Example}

%% Proof environment with a small square as a "qed" symbol
%\theoremstyle{nonumberplain}
%\theorembodyfont{\normalfont\upshape}
%\theoremsymbol{\ensuremath{\square}}
%\newtheorem{proof}{Proof}
%\newtheorem{beweis}{Beweis}


%my color shit
\definecolor{lightblue}{rgb}{0.95, 0.95, 1}
\definecolor{lightpink}{rgb}{1, 0.95, 0.95}
\definecolor{lightyellow}{rgb}{1, 0.97, 0.90}
\definecolor{lightorange}{rgb}{1, 0.95, 0.90}
\definecolor{lightgreen}{rgb}{0.95, 1, 0.90}
\definecolor{lightgrey}{rgb}{0.97, 0.97, 0.97}

\mdfsetup{skipabove=3pt, skipbelow=3pt}

\newenvironment{definition}{\begin{mdframed}[backgroundcolor=lightblue, hidealllines=true, innerleftmargin=2pt, innerrightmargin=2pt]\begin{defi}}{\end{defi}\end{mdframed}}

\newenvironment{theorem}{\begin{mdframed}[backgroundcolor=lightpink, hidealllines=true, innerleftmargin=2pt, innerrightmargin=2pt]\begin{theoremm}}{\end{theoremm}\end{mdframed}}

\newenvironment{remark}{\begin{mdframed}[backgroundcolor=lightyellow, hidealllines=true, innerleftmargin=2pt, innerrightmargin=2pt]\begin{rem}}{\end{rem}\end{mdframed}}

\newenvironment{corollary}{\begin{mdframed}[backgroundcolor=lightorange, hidealllines=true, innerleftmargin=2pt, innerrightmargin=2pt]\begin{coro}}{\end{coro}\end{mdframed}}

\newenvironment{example}{\begin{mdframed}[backgroundcolor=lightgreen, hidealllines=true, innerleftmargin=2pt, innerrightmargin=2pt]\begin{examp}}{\end{examp}\end{mdframed}}

\newenvironment{concept}{\begin{mdframed}[backgroundcolor=lightgrey, hidealllines=true, innerleftmargin=2pt, innerrightmargin=2pt]\begin{conc}}{\end{conc}\end{mdframed}}

%% Helpful macros.
%% Custom commands
%% ===============

%% Special characters for number sets, e.g. real or complex numbers.
\newcommand{\C}{\mathbb{C}}
\newcommand{\K}{\mathbb{K}}
\newcommand{\N}{\mathbb{N}}
\newcommand{\Q}{\mathbb{Q}}
\newcommand{\R}{\mathbb{R}}
\newcommand{\Z}{\mathbb{Z}}
\newcommand{\X}{\mathbb{X}}

%% Fixed/scaling delimiter examples (see mathtools documentation)
\DeclarePairedDelimiter\abs{\lvert}{\rvert}
\DeclarePairedDelimiter\norm{\lVert}{\rVert}

%% Use the alternative epsilon per default and define the old one as \oldepsilon
\let\oldepsilon\epsilon
\renewcommand{\epsilon}{\ensuremath\varepsilon}

%% Also set the alternate phi as default.
\let\oldphi\phi
\renewcommand{\phi}{\ensuremath{\varphi}}



% Own macros

\newcommand{\scalprod}[1]{\langle #1 \rangle}

\newcommand{\F}{\mathcal{F}}
\newcommand{\M}{\mathbb{M}}
\newcommand{\bigO}{\mathcal{O}}
\newcommand{\p}{\mathcal{P}}
\newcommand{\nnz}{\mathbf{nnz}}

\newcommand{\argmin}{\mathrm{argmin}}

\newcommand{\rank}{\mathrm{rank}}
\newcommand{\lsq}{\mathrm{lsq}}
\newcommand{\im}{\mathrm{Im}}
\newcommand{\Ker}{\mathrm{Ker}}


\newcommand{\D}{\mathrm{d}}
\newcommand{\Dx}{\mathrm{d}x}
\newcommand{\Dt}{\mathrm{d}t}
\newcommand{\Sr}{\mathscr{S}}

\newcommand{\fhat}{{}^{\wedge}}
\newcommand{\fcheck}{{}^{\vee}}

%% Make document internal hyperlinks wherever possible. (TOC, references)
%% This MUST be loaded after varioref, which is loaded in 'extrapackages'
%% above.  We just load it last to be safe.
\usepackage[linkcolor=black,colorlinks=true,citecolor=black,filecolor=black]{hyperref}

%Boxes
\usepackage{boxedminipage}



%% Document information
%% ===================

\title{Zusammenfassung Analysis II\\D-INFK}
% Add your names!
\author{Autor: Florian Buetler, Lasse Meinen, Marc Widmer}
\thesistype{ETH Zurich}
%\thesistype{Summary}
%\advisors{Advisors: Prof.\ Dr.\ A. D. Visor, Dr.\ P. Ostdoc}
%\department{ETH Zurich}
\date{FS18}

\begin{document}

\frontmatter

%% Title page is autogenerated from document information above.  DO
%% NOT CHANGE.

%\begin{titlingpage}
%  \calccentering{\unitlength}
%  \begin{adjustwidth*}{\unitlength-0pt}{-\unitlength-80pt}
%    \maketitle
%  \end{adjustwidth*}
%\end{titlingpage}

\fontsize{10}{11}\selectfont

%Set \tab command
\newcommand\tab[1][0.7cm]{\hspace*{#1}}

%% TOC with the proper setup, do not change.
\cleartorecto
\tableofcontents
%\mainmatter

\fontsize{9}{11}\selectfont

\mainmatter
\newpage
\twocolumn
\setcounter{page}{1}
\pagestyle{plain}

%% Your real content
\chapter{Tables \& Co}
  \begin{example}[Ableitungen und Stammfunktionen]
  	\renewcommand{\arraystretch}{1.5}
    \begin{tabular}{| c | c | c |}
          \hline 
      $f(x) = $ & $f'(x) = $ & $F(x) = $\\ %header
          \hhline{|=|=|=|}
      $c$ & $0$ & $c\cdot x$\\
          \hline
      $x^n$ & $n\cdot x^{n-1}$ & $\frac{1}{x^n}$\\
          \hline
      $\frac{1}{x}$ & $-\frac{1}{x^2}$ & $\ln{|x|}$\\
          \hline
      $\ln{|x|}$ & $\frac{1}{x}$ & $x \cdot (\ln{|x|}-1)$\\
          \hline
      $\log_a{|x|}$ & $\frac{1}{x \cdot \ln{|a|}}$ & $x \cdot (\log_a{|x|}-\frac{1}{ln{|a|}})$\\
          \hline
      $a^x$ & $\ln a\cdot a^x$ & $a^x \cdot \frac{1}{|a|}$\\
          \hline
      $\sin x$ & $\cos x$ & $-\cos x$\\
          \hline
      $\cos x$ & $-\sin x$ & $\sin x$\\
          \hline
      $\tan x$ & $1 + \tan^2 x = \frac{1}{\cos^2 x}$ & $-\ln|\cos x|$\\
          \hline
      $\sin^2 x$ & $2\sin x\cos x$ & $\frac{1}{2}(x-\sin x\cos x)$\\
          \hline
      $\cos^2 x$ & $-2\sin x\cos x$ & $\frac{1}{2}(x+\sin x\cos x)$\\
          \hline
      $\tan^2 x$ & $2 \cdot \frac{tan x}{\cos^2 x}$ & $t\tan x - x$\\
          \hline
      $\arcsin x$ & $\frac{1}{\sqrt{1-x^2}}$ & $x \cdot \arcsin{x} + \sqrt{1-x^2}$\\
          \hline
      $\arccos x$ & $-\frac{1}{\sqrt{1-x^2}}$ & $x \cdot \arccos{x} - \sqrt{1-x^2}$\\
          \hline
      $\arctan x$ & $\frac{1}{1-x^2}$ & $x \cdot \arctan{x} + \frac{1}{2}\ln{(1+x^2)}$\\
          \hline
      $\sinh x$ & $\cosh x$ & $\cosh x$\\
          \hline
      $\cosh x$ & $\sinh x$ & $\sinh x$\\
          \hline
      $\tanh x$ & $1 - \tanh^2 x = \frac{1}{\cosh^2 x}$ & $\ln{(\cosh x)}$\\
          \hline
    \end{tabular}
  \end{example}
  
  \begin{example}[Wichtige Reihen \& Limits] $ $\\
    \begin{itemize}
      \item geometrische Reihe: $\sum_{k=0}^{\infty} q^k$ = $\dfrac{1}{1-q}$ ist konvergent für $|q| < 1$, da $\sum_{k=0}^{n} q^k$ = $\dfrac{1-q^{n+1}}{1-q}$
      \item harmonische Reihe: $\sum_{k=1}^{\infty} \dfrac{1}{k}$ ist divergent
      \item alternierende harmonische Reihe: $\sum_{k=1}^{\infty} \dfrac{(-1)^{k-1}}{k}$ ist konvergent, aber nicht absolut konvergent
      \item Leibnizreihen haben die Form $\sum_{k=1}^{\infty} (-1)^{k-1}a_k$ und sind konvergent
      \item $\sum_{k=0}^{\infty} \dfrac{1}{k^j}$ ist konvergent für $j \geq 2$, keine Aussagen über $1<j<2$
      \item Euler-Mascheroni Konstante $\lim_{n \to \infty} \big(\sum_{k=1}^n \frac{1}{k} - \ln n \big)$
    \end{itemize}
  \end{example}
  
  \begin{example}[Infinite Series]
  $ $\\
  	\renewcommand{\arraystretch}{1.5}
  	\begin{tabular}{| l | l |}
  			\hline
    	$\sum_{n=0}^\infty (k+1) \cdot q^n + \frac{1}{(1-q)^2}, \; |q| < 1$ & $\sum_{n=0}^\infty a \cdot q^n + \frac{a}{1-q}, \; |q| < 1$ \\
        	\hline
        $\sum_{n=0}^\infty \frac{(-1)^k}{2k+1} = \pi /4$ & $\sum_{n=1}^\infty \frac{(-1)^{k+1}}{k} = \ln 2$ \\
        	\hline
        $\sum_{n=1}^\infty \frac{1}{k^2} = \pi^2 / 6$ & $\sum_{n=1}^\infty \frac{(-1)^{k+1}}{k} = \pi^2 / 12$ \\
        	\hline
  	\end{tabular}
  \end{example}
  
  \begin{example}[Other Important Stuff] $ $
  	\begin{itemize}
        \item $\sum_{k=1}^{n}k = \frac{n(n+1)}{2}$
		\item $\sum_{k=1}^{n}k^2 = \frac{n(n+1)(2n+1)}{6}$
		\item $\sum_{k=1}^{n}k^3 = (\frac{n(n+1)}{2})^2= (\sum_{k=1}^{n}k)^2$
		\item $\sum_{k=1}^{n}(2k-1) = n^2$
		\item $\sum_{k=1}^{n}(2k-1)^2 = \frac{n(2n-1)(2n+1)}{3}$
        \item $ \sin^2 x + cos^2x = 1 $
		\item $ \sin x+y = \sin x \cdot \cos y + \cos x \cdot \sin y $
		\item $ \cos x+y = \cos x \cdot \cos y - \sin x \cdot \sin y $
        \item $ \sin x = \frac{\exp{(ix)} - \exp {(-ix)}}{2i} $
        \item $ \cos x = \frac{\exp{(ix)} + \exp{(-ix)}}{2} $
        \item $ \sinh{x} = \frac{e^x - e^{-x}}{2} $
        \item $ \cosh{x} = \frac{e^x + e^{-x}}{2} $
        \item $ \sin^2{x} = \frac{1 - \cos{(2x)}}{2} $
        \item $ \cos^2{x} = \frac{1 + \cos{(2x)}}{2} $
        \item $ \tan^2{x} = \frac{1 - \cos{(2x)}}{1 + \cos{(2x)}} $
        \item $ \sin(\arccos x) = \sqrt{1-x^2} $ \tab $ \sin(\arctan x) = x/\sqrt{1+x^2} $
        \item $ \cos(\arcsin x) = \sqrt{1-x^2} $ \tab $ \cos(\arctan x) = 1 / \sqrt{1+x^2} $
        \item $ \tan(\arcsin x) = x/\sqrt{1-x^2} $ \tab $ \tan(\arccos x) = \sqrt{1-x^2}/x $
  	\end{itemize}
  \end{example}
  
  \begin{example} $ $\\
  	\renewcommand{\arraystretch}{1.2}
  	\begin{tabular}{| c | c || c | c | c |}
  			\hline
        Radian & Gradian & $\sin$ & $\cos$ & $\tan$ \\
        	\hline
        $0\deg$ & $0$ & $0$ & $1$ & $0$ \\
        	\hline
        $30\deg$ & $\pi/6$ & $1/2$ & $\sqrt{3}/2$ & $\sqrt{3}/3$ \\
        	\hline
        $45\deg$ & $\pi/4$ & $\sqrt{2}/2$ & $\sqrt{2}/2$ & $1$ \\
        	\hline
        $60\deg$ & $\pi/3$ & $\sqrt{3}/2$ & $1/2$ & $\sqrt{3}$ \\
        	\hline
        $90\deg$ & $\pi/2$ & $1$ & $0$ & $\infty$ \\
        	\hline
        $120\deg$ & $2\pi/3$ & $\sqrt{3}/2$ & $-1/2$ & $-\sqrt{3}$ \\
        	\hline
        $135\deg$ & $3\pi/4$ & $\sqrt{2}/2$ & $-\sqrt{2}/2$ & $-1$ \\
        	\hline
        $150\deg$ & $5\pi/6$ & $1/2$ & $-\sqrt{3}/2$ & $-\sqrt{3}/3$ \\
        	\hline
        $180\deg$ & $\pi$ & $0$ & $-1$ & $0$ \\
        	\hline
  	\end{tabular}
  \end{example}

    \section{Betrag}
	
	$|ab|=|a||b|$\\
    $|\frac{a}{b}|=\frac{|a|}{|b|}$\\
	$|a+b|\leq |a|+|b|$
	
	
	\section{Potenzen und Wurzeln}
    
	Definition: $x= \sqrt[n]{a} \Leftrightarrow (x^n = a \ und \ x\geq 0)$\\
	Es folgt:$\sqrt[n]{-a}=-\sqrt[n]{a}, a \geq 0$\\
	$a^{-n} = \frac{1}{a^n}= (\frac{1}{a})^n$\\
	$a^{\frac{1}{a}}= \sqrt[n]{a}$ \tab $\sqrt{ab}=\sqrt{a}\sqrt{b}$\\
	$a^{\frac{m}{n}}= \sqrt[n]{a^m}$\tab $\sqrt{\frac{a}{b}}= \frac{\sqrt{a}}{\sqrt{b}}$\\
	$a^x= e^{x\cdot \ln a}$ \tab $\sqrt[n]{a^{-m}}= \frac{1}{\sqrt[n]{a^m}}$
	
\subsection{Potenzgesetzte und Wurzelgesetze}
    
	$a^m a^n = a^{m+n}$ \tab $\sqrt[n]{a^m} = \sqrt[kn]{a^{km}}$
	
	\noindent$\frac{a^m}{ a^n} = a^{m-n}$ \tab $\sqrt[n]{\sqrt[k]{a}} = \sqrt[nk]{a^{km}}$
	
	\noindent$(a^m)^n= a^{mn}$\tab $\sqrt[n]{a}\sqrt[n]{b}=\sqrt[n]{ab}$
	
	\noindent$a^n b^n = (ab)^n$ \tab $\frac{\sqrt[n]{a}}{\sqrt[n]{b}}= \sqrt[n]{\frac{a}{b}}$
	
	\noindent$\frac{a^n}{b^n} = (\frac{a}{b})^n$\\


	
    \section{Logarithmensätze}
    
	$\log(uv) = \log u +\log v$ \hspace{0.05cm} $\log(\frac{u}{v}) = \log u -\log v$
	
	\noindent$\log(u^r) = r\cdot \log u$ \tab $\log(\frac{1}{v}) = -\log v$\\

\vfill

\chapter{Ordinary Differential Equations}

\begin{theorem}[Local existence of unique sol]
	Suppose $F: \mathbb{R}^3 \rightarrow \mathbb{R}$ diffbar, $x_0 \in \mathbb{R}, \  (y_0,y_0') \in \mathbb{R}$.
	Then the ODE $F(x,y,y') = 0$ has a unique sol $f$ on "largest' open interval $I$ containing $x_0$ st $f(x_0)= y_0, \ f'(x_0)=y_0'$.
\end{theorem}

\begin{definition}[Linear differential equations]
	Let $I \subset \mathbb{R}$ be an open interval and $k \geq 1$. 
	A \textbf{homogenous} linear ODE is defined as
	$$ y^{(k)} + a_{k-1}y^{(k-1)} + ... + a_1y' + a_0y = 0 $$
	, where $a_i$ are compex functions on $I$.
	A \textbf{inhomogenous} linear ODE is defined as
	$$ y^{(k)} + a_{k-1}y^{(k-1)} + ... + a_1y' + a_0y = b$$
	, where $b$ is a compex funtion on $I$.
\end{definition}

\begin{theorem}
	Consider a linear ODE. The following holds:
	\begin{itemize}
		\item Set $S$ of $k$-diffbar sols to the ODE is a subspace of all complex functions defined on $I$.
		\item $dim(S)=k$and $\forall x_0 \in I \ \forall (y_0,...,y_{k-1}) \in C^k$ exists a \textbf{unique} $f\in S$ st $f(x_0) = y_0, \ ..., \ f^{(k-1)} = y_{k-1}$.
		\item If $b \in C^0(I)$ is the inhom part of a ODE, then there exists a sol $f_0$ and $S_b$ is the set of solutions which have the form $f + f_0, \ f \in S$.
		\item $\forall x_0 \in I \ \forall (y_0,...,y_{k-1}) \in C^k$ there a \textbf{unique} sol $f \in S_b$ st $f(x_0) = y_0, \ ..., \ f^{(k-1)} = y_{k-1}$.
	\end{itemize}
	\textbf{Remark:} If $b \neq 0$, $S_b$ is not a vector space. 
\end{theorem}

\begin{concept}[Solving first order ODEs]
	$ $\\
	\begin{enumerate}
		\item \textbf{Solve homogenous part:}
			Formally, transform $y' + ay = 0'$ into $y'/y = -a$ which implies $(log|y|)' = -a$.
			Integration gives us
			$$ y = z\exp(-A) $$
			, where $z \in C$ and $A$ is primitive of $a$.
		\item \textbf{Solve inhomogenous part:}
			Consider $y' + ay = b$.
			It sufficies to find a single sol $f_0$.
			If no obvious sol comes to mind, use \textbf{variation of constant}:
			Start with the homogenous sol $f(x) = z\exp(-A(x))$, where $z$ is now a function.
			Form this we can deduce that
			$$z'(x) = b(x)\exp(A(x))$$.
			You now have an inhom sol: $f_0(x) = C(x)\exp(-A(x))$
			, where $C(x)$ is prmitive of $z$
		\item \textbf{Combine}
			You can now obtain all solutions $f + f_0$.
	\end{enumerate}
\end{concept}

\begin{concept}[Solving ODEs with constant coeffs]
	\begin{enumerate}
		\item \textbf{Find homogenous solution (Euler-Ansatz)}
			Replace all $y(x)$ with $e^{\lambda x}$ and derive all the resulting derivatives.
			Then, divide by $e^{\lambda x}$. This gives you the \textbf{characteristic polynomial}.
			Now compute all roots $\lambda_i$ and their multiplicity $m_i$.
			From this, you can create a  base sol-space: $B = {x^{m_i-1}e^{\lambda x} | i \in \N, \ i \leq k}$.
		\item \textbf{Find inhomogenous solution}
			(Considering $y''+a_1y'(x) = g(x)$)
			You are looking for a sol of the form: $y_p(x) = C_1(x)y_q(x) + C_2(x)y_2$, with $y_i(x)$ from Euler Ansatz and $C_i(x)$ satisfying the system 
			$$\begin{bmatrix} y_1(x) & y_2(x) \\ y'_1(x) &  y'_2(x)\end{bmatrix} \cdot \begin{bmatrix}C_1(x) \\  C_2(x)\end{bmatrix} = \begin{bmatrix}0 \\ g(x)\end{bmatrix}$$
			We check that the determinand of the system matrix doesn't vanish, meaning that a unique sol exists.
			We find the coefficients by using one of two options:	
			\begin{enumerate}
				\item Inversion of the matrix and then integrating:
					$$\begin{bmatrix} C'_1(x) \\ C'_2(x)\end{bmatrix} = $$ $$\frac{1}{y_1(x)y'_2(x) - y_2(x)y'_1(x)}\begin{bmatrix} y_1(x) & y_2(x) \\ y'_1(x) & y'_2(x) \end{bmatrix} \cdot \begin{bmatrix}C'_1(x) \\ C'_2(x) \end{bmatrix} = \begin{bmatrix} 0 \\ g(x) \end{bmatrix}$$
				\item or by directly using
					$$y_p(x) = -y_1(x) \int \frac{y_2(x)g(x)}{y_1(x)y'_2(x) - y_2(x)y'_1(x)}dx $$ $$ y_2(x) \frac{y_1(x)g(x)}{y_1(x)y'_2(x) - y_2(x)y'_1(x)}dx$$
			\end{enumerate}
		\item The general solutions are obtained by $y(x) = y_h(x) + y_p(x)$.
	\end{enumerate}
\end{concept}

\begin{remark}[Trick for solving ODEs]
	Sometimes it might be useful to replace $f(x)$ with $h(x) = f(g(x))$ if it simplifies the equation.
	The obtained solution can easily be converted to a solution of $f$.
\end{remark}

% Add table of Ansaetze

%\pagebreak
\chapter{Die reellen Zahlen}
%\begin{multicols}{3}

%% Füllt die Lücken. :)
%% Ihr habt einige Freiheiten dabei, z.B. dürft ihr gerne zusätzliche Dinge hinzufügen etc.
%% Einige Dinge sind vordefiniert (dank eines Freundes von mir), aber wenn ihr andere Formatierungen verwenden wollt, könnt ihr das gerne tun.
%% Bitte fragt mich vorher, wenn ihr größere Änderungen an den config-Dateien vornehmen wollt.



\section{Elementare Zahlen}
\begin{theorem}
Jede reelle Zahle lässt sich beliebig gut durch rationale Zahlen approximieren.
$$\forall x\in \R \ \forall \epsilon . 0 \exists q \in \Q: |x-q| \leq \epsilon$$
\end{theorem}

\section{Die Axiome der reellen Zahlen}
	\begin{description}
    
	\item{Addition}
    	
	\begin{description}
    \setlength\itemsep{0em} %% Kompakter als vorher -- Am besten irgendwo in die Configs einfügen
	
		\item{A-1} Assoziativität: $\forall x,y,z\in\R:x+(y+z) = (x+y)+z$
		\item{A-2} Neutrales Element: $\exists0\in\R, \forall x\in\R:x+0=x$
		\item{A-3} Inverses Element:$\forall x \in\R, \exists y \in\R: x+y=0$
		\item{A-4} Kommutativität: $\forall x,y \in\R: x+y = y+x$
	\end{description}
    
    \item{Multiplikation}
	\begin{description}
        \setlength\itemsep{0em} %% Kompakter als vorher -- Am besten irgendwo in die Configs einfügen
	
		\item{M-1} Assoziativität: $\forall x,y,z\in\R:x\cdot (y\cdot z) = (x\cdot y)\cdot z$
		\item{M-2} Neutrales Element: $\exists 1\in\R, \forall x\in\R:x\cdot 1=x$
		\item{M-3} Inverses Element:$\forall x \in\R, \exists y \in\R: x\cdot y=1$
		\item{M-4} Kommutativität: $\forall x,y \in\R: x\cdot y = y\cdot x$
	\end{description}
    
    \item{Distributiv-Gesetz}
	\begin{description}
    \setlength\itemsep{0em} %% Kompakter als vorher -- Am besten irgendwo in die Configs einfügen
	
		\item{D} $\forall x,y,z\in\R:x\cdot (y+z) = x\cdot y + x\cdot z$
	\end{description}
    
    \item{Ordnung}
	\begin{description}
		\setlength\itemsep{0em} %% Kompakter als vorher -- Am besten irgendwo in die Configs einfügen
	
    	\item{O-1} Reflexivität: $\forall x\in\R: x \leq x$
		\item{O-2} Transitivität: $\forall x,y,z\in\R: x\leq y \wedge y \leq z \implies x \leq z$
		\item{O-3} Identitivität:$\forall x,y \in\R: x \leq y \wedge y \leq x \implies x = y$
		\item{O-4} Die Ordnung ist total: $\forall x,y \in\R: x\leq y$ oder $y\leq x$
	\end{description}
    
    \item{Konsistenz}
	\begin{description}
    	\setlength\itemsep{0em} %% Kompakter als vorher -- Am besten irgendwo in die Configs einfügen
	
		\item{K-1} $\forall x,y,z\in\R: x\leq y \implies x+z \leq y+z$
        \item{K-2} $\forall x,y,z\in\R: x\leq y \implies x*z \leq y*z$
	\end{description}
    
    \item{Ordnungsvollständigkeit}
	\begin{description}
		\item{V} Für je zwei nicht leeren Mengen $A,B\subset\R$ mit 
        $$\forall a\in A, \forall b\in B: a\leq b$$
        gibt es ein $c\in\R$ sodass gilt
        $$\forall a\in A, \forall b\in B: a\leq c\leq b$$
	\end{description}
    
    \item{\underline{Folgerungen der Axiomen:}} 
    \begin{enumerate}
    \item $\forall x \in \mathbb{R}$: $(-1)\cdot x = -x$
	\item $(-1)\cdot (-1)=1$
	\item $\forall x \in \mathbb{R}: x^2 \geq 0$
	\item $0 < 1 < 2 < ...$
	\item $\forall x > 0: x^{-1} > 0$
	\item $\forall x,y \geq 0: x \leq y \Leftrightarrow x^2 \leq y^2$
	\item  Es gibt $c\in \mathbb{R}$ mit $c^2 = 2$
	\item  $x\leq |x|, \forall x \in X$
	\item  $|xy|=|x||y|, \forall x,y \in \mathbb{R}$
    
    
    \end{enumerate}
    
\end{description}

\begin{theorem}[Young]
Für $x,y \in R$, $ \epsilon > 0$ gilt, \tab
$$
2|x\cdot y| \leq \epsilon x^2 + \frac{1}{\epsilon}y^2
$$
\end{theorem}

\begin{theorem}[Bernoullische Ungleichung]
Für $x>-1$ und $n\in \N$ gilt
$$(1+x)^n \geq nx$$
\end{theorem}

% Archimedisches Prinzip
\begin{definition}[Archimedisches Prinzip]
Die natürlichen Zahlen sind nach oben hin nicht beschränkt:
	$$ \forall 0 < b\in \R, \exists n \in \N: b < n.$$
\end{definition}

\begin{theorem}[Fundamentalsatz der Algebra]
Jedes Polynom von Grad $n\geq q$ hat in $\C$ eine Nullstelle.
\end{theorem}

\section{Supremum und Infimum}
 %% Ersetz mich mit der Definition des Supremums. Bitte inklusive B2) aus der Übung.
 	\begin{definition}[obere / untere Schranke]
    Sei $X \subset \mathbb{R}$.
    Eine reelle Zahl $y$ mit $\forall x \in X: x \leq y (x\geq y)$ heisst \textbf{obere (untere) Schranke} von X.
    Existiert eine solche Schranke, so ist X \textbf{nach oben (unten) beschränkt}.
    \end{definition}
 
 	\begin{definition}[Supremum / Infimum]
    Jede nicht leere und nach oben beschränkte Teilmenge $X \subset \mathbb{R}$ besitzt eine eindeutige kleinste obere Schranke, genannt Supremum.
    (Infimum analog)
    
	Für eine nach oben (unten) unbeschränkte Teilmenge $X \subset \mathbb{R}$ gilt: $sup(X) = \infty$ ($inf(X) = -\infty$).
    
    Ist $sup(X) \in X$ so wird das \textbf{Supremum in X angenommen}.
	\end{definition}
    
    \begin{concept}[Sup/Inf]
        \item{A) Obere Schranke}

    			\item{} Seien $A, S\subset\R$ nicht leere Mengen mit
    			$$\forall a\in A,s\in S: a\leq s$$ 
				dann ist S die Menge aller oberen Schranken von A.

    \item{B) Kleinste obere Schranke}

    		\item{B1)} Supremum von A oder Sup(A) ist definiert als:
    			$$\forall o \in S, s:= Sup(A): s \leq o$$ 
    		\item{B2)} Alternativ auch: 
    			$$\forall \varepsilon > 0,\exists a\in A:a > s - \varepsilon$$
	\end{concept}
    
    \begin{theorem} [Sup/Inf beschränkter Mengen]
    Jede nach oben/unten beschränkte Teilmenge besitzt ein Supremum/Infimum.
    \end{theorem}

\noindent\begin{boxedminipage}{225pt}
		\underline{\textbf{Rezept:}} \textit{Supremum zeigen}
		
 		Sei A eine Menge\\
		\textbf{1.} "Rate" $supA = s$\\
		\textbf{2.} Zeige:\ $\forall a \in A, a \leq s$\newline
		\textbf{3.} Zeige entweder:\ $\forall t \textrm{ obere Schranken}, s \ \leq t$\newline
		\tab \tab oder:\  $\forall \varepsilon \textgreater 0 \quad \exists a \in A, |s-\varepsilon| \textless a$
		
	\end{boxedminipage}\\

%\end{multicols}

%\chapter{Folgen und Reihen}

%% Füllt die Lücken. :)
%% Ihr habt einige Freiheiten dabei, z.B. dürft ihr gerne zusätzliche Dinge hinzufügen etc.
%% Einige Dinge sind vordefiniert (dank eines Freundes von mir), aber wenn ihr andere Formatierungen verwenden wollt, könnt ihr das gerne tun.
%% Bitte fragt mich vorher, wenn ihr größere Änderungen an den config-Dateien vornehmen wollt.

\begin{definition}[Folgen]
	Eine Abbildung $\mathbb{N}_0 \rightarrow \mathbb{R}, k \mapsto a_k$ heisst \textbf{(unendliche) Folge} in $\mathbb{R}$.
    Eine Folge heisst \textbf{beschränkt}, falls die Teilmenge $\{a_n : n \geq 1\} \subset \mathbb{R}$ beschränkt ist.
\end{definition}

\begin{example}
\textbf{Folge definiert durch Formel:}
	\begin{itemize}
	\item $a_n = 1$ , oder
    \item $a_n = n$ , oder
    \item $a_n = \frac{1}{1+n^2}$
	\end{itemize}
	
	\noindent\textbf{Folge rekursiv definiert:}
	$a_n =$ (Funktion definiert über $a_{n-1},a_{n-2},...,n$)\newline
	geg. $(a_n) n \in \mathbb{N} \quad auf \quad\mathbb{R}$\newline
	\tab $s_1 = a_1$\newline
	\tab $s_{n+1} = a_{n+1}+s_n$\\
	
	\noindent Wichtige Folgen:
	
	\noindent\textbf{Arithmetische Folge:}\\
	$a_n$ , sodass $a_{n+1}-a_n$ konstant. D.h. $a_{n+1}-a_n = d (\in \mathbb{C})$\newline
	 $\Rightarrow a_1 = a$\newline
	\hspace*{0.32cm} $a_2= a_1 +d = a+d$\newline
	\hspace*{0.32cm} $a_3 = a+2\cdot d,...\Rightarrow \underline{a_n = a+(n-1)d}$
	
	\noindent\textbf{Geometrische Folge:}\\
	$a_n$ , sodass $q \in \mathbb{C}, a_{n+1} = q\cdot a_n$
	
	\noindent$\Rightarrow a_1 = a$\newline
	\hspace*{0.32cm} $a_2= qa$\newline
	\hspace*{0.32cm} $a_3 = q^2a$\newline
	\tab ...\newline
	\hspace*{0.32cm} \underline{$a_n = q^{n-1}a$}
\end{example}

\section{Grenzwert einer Folge}
	Sei $(a_n)_{n\in \mathbb{N}}=(a_1,a_2,a_3...)$ einer Folge in $\mathbb{R}, a \in \mathbb{R}$\\
	
	\begin{definition}[Konvergenz]
    Die Folge $(a_n)_{n\in \mathbb{N}}$\textbf{ konvergiert} gegen $a$ für $n \rightarrow \infty$, falls gilt:
	
		$$\forall \varepsilon > 0 \ \exists n_0\in \mathbb{N} \ \forall n \geq n_0:|a_n|-a < \varepsilon$$
		
		\raggedright{Falls ein solcher \textbf{Grenzwert / Limes} existiert, so schreiben wir}
		
		$$a = \lim_{x \to \infty}  a_n$$
        oder $a_n \rightarrow a (n\rightarrow \infty)$.
        Ansonsten ist die Folge \textbf{divergent}.
	\end{definition}
	
    \begin{theorem}
    	Der Grenzwert einer konvergierenden Folge ist eindeutig bestimmt.
    \end{theorem}
    
\section{Konvergenzkriterien}
	
	\begin{definition}[Monotone Konvergenz]
    	Eine monotone reelle Folge $(a_n)_n$ konvergiert gdw. sie beschränkt ist.\\
        Es gilt: $\lim_{n\to\infty} a_n = \sup a_n$
	\end{definition}
	
    	\noindent\begin{definition}[Cauchy-Folge]
	\noindent $(a_n)_{n\in \mathbb{N}}$ heisst \textbf{Cauchy-Folge}, falls gilt:
		$$\forall \varepsilon > 0\ \exists n_o = n_0(\varepsilon) \in \mathbb{N}\ \forall n,l \geq n_0 : |a_n-a_l| < \varepsilon$$
       	Eine Folge $(a_n)_n \in \C$ ($(a_n)_n = (b_n)_n + i(c_n)_n)$) ist eine Cauchy-Folge, gdw. $(a_n)_n$ \& $(b_n)_n$ Cauchy-Folgen sind.
	\end{definition}

\begin{theorem}[Cauchy-Kriterium]

Für $(a_n)_{n\in \N} \subset \R$ gilt:

$(a_n)_{n\in \mathbb{N}}$ ist konvergent $\Longleftrightarrow$
$(a_n)_{n\in \mathbb{N}}$ ist Cauchy-Folge.
\end{theorem}

\begin{theorem}[Sandwich Satz]
	Seien $(a_n)_n$, $(b_n)_n$, $(c_n)_n$ drei Folgen, sodass $a_n \leq b_n \leq c_n$ für alle $n \in \mathbb{N}$.
    Angenommen $(a_n)_n$ und $(c_n)_n$ sind konvergent und $a = \lim a_n = \lim c_n$, dann ist auch $(b_n)_n$ konvergent und $\lim b_n = a$.
\end{theorem}
    
	\begin{theorem}[Rechnen mit Folgen]
    Sind Folgen $(a_n)_{n\in \mathbb{N}},(b_n)_{n\in \mathbb{N}} \subset \mathbb{R}\ $ konvergent mit $\lim_{n \to \infty} a_n= a, \lim_{n \to \infty}b_n = b$.
    Dann konvergieren die Folgen $(a_n+b_n)_{n\in \mathbb{N}}, (a_n\cdot b_n)_{n \in \mathbb{N}}$ und:
	
	\textbf{i)} $\lim_{n \to \infty} (a_n+b_n)= a+b = \lim_{n \to \infty}a_n+\lim_{n \to \infty}b_n$
	
	\textbf{ii)} $\lim_{n \to \infty} (a_n\cdot b_n)= a\cdot b = \lim_{n \to \infty}a_n\cdot\lim_{n \to \infty}b_n$
	
	\textbf{iii)} Falls zusätzlich $b\neq 0 \neq b_n$ für alle $n$, so gilt $\lim_{n \to \infty}(\frac{a_n}{b_n})= \frac{a}{b}$
	
	
	\textbf{iv)} Falls $a_n \leq b_n \forall n\in \mathbb{N}$, so auch $a\leq b$
	
	\end{theorem}

\section{Teilfolgen, Häufungspunkte}
\begin{definition}[Teilfolge]
Sei $l:\N \rightarrow \N$ eine injektive, monoton wachsende Funktion. (Glieder können ausgelassen werden.)

Dann heißt $(a_{l(n)})_{n\in \N}$ eine Teilfolge von $(a_n)_n$.
\end{definition}

\begin{definition}[Häufiungspunkt]
Für eine Folge $(a_n)_n\in\R$ heißt $a\in\R$ ein Häufungspunkt, wenn eine Teilfolge von $(a_n)_n$ gegen a konvergiert.
\end{definition}

\begin{theorem}[Bolzano-Weiherstrass]
Jede beschränkte Folge hat eine konvergente Teilfolge und damit auch einen Häufungspunkt.
\\
Weiter gilt:
\begin{itemize}
\item Konvergiert eine Folge, so ist die grösste konv. TF gleich der kleinsten konv. TF.
\item Konvergiert eine Folge gegen $a$, so konvergiert jede TF gegen $a$.
\end{itemize}

\end{theorem}

% \section{Folgen in $\R ^d$ oder $\C$}


\section{Reihen}

\begin{definition}[Reihen]
	Sei $(a_k)_k$ eine Folge in $\mathbb{R}$ oder $\mathbb{C}$.
	Ein Ausdruck der Form $\sum_{k=0}^{\infty}a_k$ heisst eine \textbf{(unendliche) Reihe}.
    
    Wir definieren zusätzlich die \textbf{Folge von Partialsummen} $$(S_n)_n = (\sum_{k=0}^{n} a_k)_n$$
\end{definition}

\begin{definition}[Konvergenz einer Reihe]
Die Reihe $\sum_{k=1}^{\infty}a_k$ ist konvergent, falls die Folge von Partialsummen konvergiert.
D.h.
$$\lim\limits_{n \rightarrow \infty} S_n = \lim\limits_{n \rightarrow \infty} \sum_{k=1}^{n} a_k = \sum_{k=1}^{\infty}a_k$$
existiert.
\end{definition}

\begin{theorem}[Cauchy-Kriterium]
$\sum_{k=1}^{\infty} a_k$ ist konvergent $\Leftrightarrow \forall \epsilon > 0$ $\exists N \in \mathbb{N}$ s.d. $$\forall n \geq N, m\in \mathbb{N} : |\sum_{k=m+1}^{n+m} a_k| < \epsilon$$

d.h. man kann 0 mit Partialsummen beliebig annähern
\end{theorem}

\begin{theorem}[Notwendige Bedingung für Konvergenz]
Sei $(S_n)_{n\mathbb{N}} = \sum_{k=1}^{\infty} a_k$ eine konvergente Reihe, dann gilt $lim_{n\rightarrow\infty} a_k = 0$

d.h. $lim_{n\rightarrow\infty} a_k \neq 0 \Rightarrow (S_n)$ divergent

\textbf{Gilt nicht in der anderen Richtung!}
\end{theorem}

\begin{theorem}[Rechnen mit Reihen]
Seinen $\sum_{k=1}^\infty a_k$, $\sum_{k=1}^\infty b_k$ konvergente Reihen, und $\alpha, \beta \in \mathbb{C}$, dann sind die Reihen $\sum_{k=1}^\infty (a_k + b_k)$, $\sum_{k=1}^\infty (\alpha a_k)$ konvergent und es gilt $\alpha \sum_{k=1}^\infty a_k + \beta \sum_{k=1}^\infty b_k = \sum_{k=1}^\infty (\alpha a_k + \beta b_k)$
\end{theorem}

% Es gibt hier noch einen Satz zu Potenzreihen. War der dran? Sonst muss er hier nicht rein.

\section{Absolute Konvergenz}

\begin{definition}[Absolute Konvergenz]
Die Reihe $\sum_{k=1}^{\infty}a_k$ heißt absolut konvergent, falls $\sum_{k=1}^{\infty}|a_k|$\ konvergiert.
\end{definition}
    
\begin{theorem}
 Absolute Konvergenz $\Rightarrow$ Konvergenz
\end{theorem}

\begin{theorem}[Quotientenkriterium]
Sei $\forall k \in \N: a_k\neq 0$.
\begin{enumerate}
\item Falls $\lim_{n\rightarrow \infty}|\frac{a_{n+1}}{a_n}|<1$ oder auch $\exists 0\leq C<1:|\frac{a_{n+1}}{a_n}|\leq C$ für n groß genug, so konvergiert die Reihe $\sum_{n=1}^{\infty}a_n$ absolut.
\item Falls $\lim_{n\rightarrow \infty}|\frac{a_{n+1}}{a_n}|>1$ oder auch $\exists C>1:|\frac{a_{n+1}}{a_n}|\geq C$ für n groß genug, so divergiert die Reihe $\sum_{n=1}^{\infty}a_n$.
\end{enumerate}
\end{theorem}

\begin{theorem}[Wurzelkriterium]
\begin{enumerate}
\item Falls $\lim_{n\rightarrow \infty}\sqrt[n]{|a_n|}<1$ oder auch $\exists 0\leq C<1:\sqrt[n]{|a_n|}\leq C$ für n groß genug, so konvergiert die Reihe $\sum_{n=1}^{\infty}a_n$ absolut.
\item Falls $\lim_{n\rightarrow \infty}\sqrt[n]{|a_n|}>1$ oder auch $\exists C>1:\sqrt[n]{|a_n|}\geq C$ für n groß genug, so divergiert die Reihe $\sum_{n=1}^{\infty}a_n$.
\end{enumerate}
\end{theorem}

\begin{remark}
In beiden Kriterien gilt: Ist der Limes gleich 1, so hilft uns das jeweilige Kriterium nichts und wir müssen uns andere Strategien überlegen.
\end{remark}

\begin{remark}
Quotienten- und Wurzelkriterium sind äquivalent.
\end{remark}


\begin{theorem}
% Ersetz mich durch Theorem zum Umordnen von Reihen!
Sei $ \sum_{k=1}^{\infty}a_k$  \textbf{absolut konvergent} und sei $\phi: \N \rightarrow \N $ bijektiv. Dann ist auch die "umgeordnete" Reihe $ \sum_{k=1}^{\infty}a_{\phi(k)}$ konvergent, und es gilt
\[
\sum_{k=1}^{\infty}a_{\phi(k)} = \sum_{k=1}^{\infty}a_k
\]
\end{theorem}

\begin{theorem}[Riemann]
Sei $\sum_{k=1}^{\infty}a_k$ konvergent, aber nicht absolut konvergent. Dann gilt:
$$\forall x\in \R\exists\phi:\! \sum_{k=1}^{\infty}a_{\phi(n)} = x$$,
wobei $\phi: \N \to \N$ bijektiv.
\end{theorem}

\begin{theorem}[Abs. Konv. $\Rightarrow$ Konv. des Produkts]
Seien $(a_n)_n, (B_n)_n \in \C$, $\sum_{k=1}^{\infty}a_k$, $ \sum_{k=1}^{\infty}b_k$  absolut konvergent. Dann ist auch das Produkt der beiden Reihen absolut konvergent und es gilt
$$\sum_{k=0}^{\infty} \sum_{l=0}^{\infty}{a_k b_l} = \sum_{k=0}^\infty a_k \sum_{k=0}^\infty b_k$$.
\end{theorem}

\begin{definition}
Eine \textbf{Potenzreihe} ist eine Reihe der Form $\sum_{k=0}^\infty c_k z^k$.
\end{definition}

\begin{theorem}[Leibniz-Reihe]
Sei $(a_n)_{n\in\N}$ eine monotone Folge positiver Zahlen mit $a_{n-1} \geq a_n \to 0 (n \to \infty)$, so konvergiert die Reihe 
$$S_n = \sum_{k=1}^n (-1)^k \cdot a_k$$
Es gilt die Fehlerabschätzung $|S-S_n| \leq a_{n+1}$.
\end{theorem}

\begin{definition}[Konvergenzradius]
Wir definieren den Konvergenzradius einer Potenzreihe als
$$\rho = \frac{1}{\lim_{n\rightarrow\infty} {\sqrt[n]{c_n}}},$$
falls der Grenzwert existiert.
Alternativ gilt auch $$\rho = \lim_{n\rightarrow\infty} \frac{a_n}{a_{n+1}}.$$
\end{definition}

\begin{theorem}[Konvergenz von Potenzreihen]
Aus dem Quotienten- und Wurzelkriterium folgt direkt, dass:
\begin{itemize}
\item $|x|<\rho\Rightarrow$ Die Potenzreihe ist abs. konv.
\item $|x|>\rho \Rightarrow$ Die Potenzreihe ist divergent.
\end{itemize}
Für $|x| = \rho$ ist das Konvergenzverhalten unklar!
\end{theorem}

\begin{definition}[Exponentialfunktion]
$$exp(x) = \sum_{k=0}^{\infty} \frac{x^k}{k!} = \lim_{n \rightarrow \infty} (1+\frac{x}{n})^n = e^x (\forall x \in \C)$$
\end{definition}

\begin{definition}[Logarithmus]
Wir definieren den Logarithmus als die stetige (siehe Umkehrsatz für offene Intervalle) Umkehrfunktion der Exponentialfunktion. 
$$log=(exp|_\R)^{-1} : ]0,\infty[\to\R.$$

(Wir schränken $exp$ hier auf $\R$ ein, weil der Logarithmus nur auf $\R$ eindeutig definiert ist, die Exponentialfunktion dagegen auch auf $\C$.)
%Das kann man so leider nicht schreiben. Außerdem haben wir den Logarithmus mit spezieller Basis bisher, glaube ich, gar nicht eingeführt. (MG)
%Sei $f: x \rightarrow b^{x}$ , dann ist $ f^{-1} = b^{x} \rightarrow x  = \log_{b}(x)$
\end{definition}

\begin{theorem}[Additionstheorem]
	Für alle $z, w \in \C$ gilt $exp(z+w) = exp(z) \cdot exp(w)$.
    
    Somit gilt für alle $z \in \C$, dass $exp(z) \cdot exp(-z) = exp(0) = 1$. Insbesondere folgt, dass $exp(z) \neq 0$.
\end{theorem}

\begin{remark}
Exponentialfunktion für rein imaginäre Argumente $z = iy$ ($y\in \R$), können wir Exp(iy) durch Umordnung gemäss Theorem 2.19 in Real- und Imaginärteil zerlegen
\[
Exp(iy) = \sum_{k=0}^{\infty} \frac{(iy)^k}{k!} =: \cos(y) + i\sin(y).
\]
\end{remark}

\begin{remark}
Für komplexe Zahlen $z=x+iy \in \C$ gilt
\[
exp(x+iy) = exp(x) \cdot exp(iy) = exp(x) \cdot (\cos(y)+i\sin(y)).
\]
\end{remark}

\begin{definition}[Trigonometrische Funktionen]
Wir definieren den Sinus, Cosinus und Tangens anhand von Taylorreihen:
$$sin(x) = \sum^{\infty}_{n = 0} (-1)^n \frac{x^{2n+1}}{(2n+1)!}= \frac{x}{1} - \frac{x^3}{3!}+\frac{x^5}{5!} \cdots$$
$$cos(x) = \sum^{\infty}_{n = 0} (-1)^n \frac{x^{2n}}{(2n)!}= \frac{x^0}{0!} - \frac{x^2}{2!}+\frac{x^4}{4!} \cdots$$
$$tan(x) = \frac{sin(x)}{cos(x)}$$
Der Tangens ist nicht definiert für $\{k\pi + \frac{\pi}{2} | k \in \Z\}$

% Tangens mit ausführlicher Formel? - JS - Nein, passt so. (MG)
\end{definition}
%\chapter{Stetigkeit}

\begin{definition}[Abschluss]
Sei $\Omega \subset \R ^d$.
Der \textbf{Abschluss} von $\Omega$ ist die Menge
$$\overline{\Omega} = \{x \in \R^d; \ \exists(x_k)_{k\in \mathbb{N}}: \ x_k \rightarrow x \quad (k \rightarrow \infty)\}$$

In anderen Worten, ein Intervall kann abgeschlossen werden, indem man $\sup I$ \& $\inf I$ hinzunimmt, falls sie in $\R$ liegen.

Seien $a,b \in \R$, so sind $[a,b]$, $[a,\infty[$ \& $]-\infty,b]$ abgeschlossen.

Offenbar gilt $\Omega \subset \overline{\Omega}$
\end{definition}

\begin{definition}[Kompakt]
	Ein Intervall ist \textbf{kompakt}, wenn er \textbf{abgeschlossen} und \textbf{begrenzt} ist.
    \\\\
    (Ein Intervall $K\subset \R$ heisst kompakt, wenn jede Folge in $K$ einen Häufigkeitspunkt besitzt.)
\end{definition}

\begin{definition}[Grenzwert einer Funktion]
Sei $f: \Omega \rightarrow \R^n$, $x_0 \in \bar{\Omega}$, $a \in \R^n$.
Die Funktion $f$ hat an der Stelle $x_0$ den Grenzwert $a$, falls für \textbf{jede Folge} $(x_k)_{k \in \N}$ in $\Omega$ mit $x_k \rightarrow x_0 (k \to \infty)$ gilt $f(x_k) \to a (k \rightarrow \infty)$.\\
Man schreibt: $$\lim_{x \to x_0} f(x) = a$$

Insberondere müssen der \textbf{linke} und der \textbf{rechte Grenzwert} übereinstimmen: 
$$\lim_{x \to x_0^-} f(x) = \lim_{x \to x_0^+} f(x) = \lim_{x \to x_0} f(x) = a$$
\end{definition}

\begin{definition}[Stetigkeit in einem Punkt]
Sei $f: \Omega \rightarrow \R^n $ mit $\Omega \in \R^d$ eine Funktion. \\
Die Funktion $f$ ist genau dann stetig in $\xi$, wenn für \textbf{jede gegen} $\xi$ konvergente Folge $(x_k)_{k \in \N}$, mit Elementen $x_k \in X$, die Folge $(f(x_k))_{k \in \N}$ gegen $f(\xi)$ konvergiert.\\

\textbf{Alternative Definition:} Die Funktion $f$ ist genau dann stetig in $\xi$, wenn zu jedem $\epsilon > 0$ ein $\delta > 0$ existiert, so dass für alle $x \in X$ mit $|x - \xi| < \delta$ gilt:
$$|f(x) - f(\xi)| < \epsilon$$

Die Funktion $f$ heisst an der Stelle $x_0 \in \bar{\Omega}\backslash \Omega$ \textbf{stetig ergänzbar}, falls $\lim_{x \to x_0} f(x) =: a$ existiert. (In diesem Fall ist die um den Punkt $(x_0,a)$ ergänzte Funktion offenbar stetig.)
\end{definition}

\begin{definition}[Stetigkeit einer Funktion]
Eine Funktion $f$ heisst stetig, falls sie in jedem Punkt des Definitionsbereiches stetig ist.\\
\end{definition}

\begin{theorem}
Eine monoton wachsende Funktion ist in maximal abzählbaren Punkten unstetig.
\end{theorem}

\begin{theorem}[Rechnen mit stetigen Funktionen]
Seien $f_1,f_2: \Omega \to \R$ stetig, sei $\alpha\in\R$.\\
Die Funktionen $(f_1 + f_2)$ sowie $\alpha f_1$, mit $\alpha \in \R$ beliebig, sind stetig.\\
Die stetigen Funktionen bilden einen Vektorraum.

Außerdem gilt:\\
Seien $f_1,f_2: \Omega \to \R$ stetig in $x_0\in\Omega$, sei $\alpha\in\R$.\\
Dann ist $f_1\cdot f_2$ stetig in $x_0$ und für $f_1(x_0)\neq 0$ ist auch $\frac{1}{f_1(x)}$ stetig in $x_0$.
\end{theorem}

\begin{theorem}[Stetigkeit unter Verknüpfung]
Seien $f_1:Q\to \Omega$, $f_2: \Omega \to \R$ stetig.
$\Rightarrow f_2\circ f_1$ ist ebenfalls stetig.
\end{theorem}

\begin{theorem}
Eine stetige Funktion nimmt auf einem kompakten Intervall ein Minimum und ein Maximum an.
\end{theorem}

\begin{theorem}[Zwischenwertsatz]
Seien $-\infty < a < b < \infty$ und sei $f: [a, b] \rightarrow \R$ stetig, $f(a) \leqslant f(b)$.\\
Dann gibt es zu jedem $y \in [f(a),f(b)]$ ein $x \in [a,b]$ mit $f(x) = y$.
\end{theorem}

\begin{theorem}[Umkehrsatz für kompakte Intervalle]
Sei $f:[a,b]\rightarrow\R$ stetig und streng monoton wachsend.

Dann ist das Bild von f das kompakte Intervall $[f(a),f(b)]$, $f$ ist bijektiv und $f^{-1}$ ist ebenfalls stetig.
\end{theorem}

\begin{theorem}[Umkehrsatz für offene Intervalle]
Sei $f:]a,b[\rightarrow\R$ stetig und streng monoton wachsend mit monotonen Limites: $$-\infty\leq c:=\lim_{x\searrow a}f(x) <\lim_{x\nearrow b}f(x)=:d\leq \infty$$ (Beide Limites existieren, können aber auch $\infty$ sein.)

Dann ist das Bild von f das offene Intervall $]c,d[$, $f$ ist bijektiv und $f^{-1}$ ist ebenfalls stetig.
\end{theorem}

\section{Konvergenz von Funktionenfolgen}

\begin{definition}[Punktweise Konvergenz einer Funktionenfolge]
Die Folge $(f_k)_{k\in \N}$ konvergiert punktweise gegen f,
falls gilt:
$$\forall x \in \Omega: \ f_k(x)\to f(x) \ (k \to \infty)$$

Alternativ:
$$\forall x \in \Omega \forall \epsilon > 0 \exists N \forall n > N: \ ||f_n(x) - f(x)|| < \epsilon$$
\end{definition}

\begin{definition}[Gleichmäßige Konvergenz einer Funktionenfolge]
Die Folge $(f_k)_{k\in \N}$ konvergiert gleichmässig gegen f, falls
$$ \lim_{k \to \infty} \sup_{x \in \Omega} |f_k(x) - f(x)| \to 0$$

Alternativ:
$$\forall \epsilon > 0 \exists N \in \mathbb{N}: \ \forall n > N \forall x \in \Omega: \ ||f_n(x) - f(x)|| < \epsilon$$

Die gleichmässige Konvergenz impliziert die punktweise Konvergenz.
\end{definition}

\begin{theorem}[glm. Konv. stetiger Funktionen]
Seien $f_k : \Omega \subset \R^{d} \to \R^{n}$ stetig, $k \in \N$. Weiter gelte $f_k \to f$ ($k \to \infty$) \textbf{gleichmässig} zu einem $f : \Omega \to \R^n$. Dann ist $f$ \textbf{stetig}.
\end{theorem}

\begin{corollary}
Für $|x|<\rho$ (also im Inneren des Konvergenzkreises) sind Potenzreihen stetige Funktionen.
\end{corollary}

%\chapter{Differentialrechnung auf $\R$}
\section{Differential und seine Regeln}

\begin{definition}[Differenzierbarkeit]
	Eine Funktion $f$ heisst \textbf{differenzierbar} an der Stelle $x_0$, falls der Grenzwert
	$$\lim_{x \to x_0} \frac{f(x)- f(x_0)}{x-x_0}=: f'(x_0) =: \frac{df}{dx}(x_0)$$
    
existiert.
	
	Eine Funktion $f$ heisst \textbf{differenzierbar}, falls der obige Grenzwert an jeder Stelle des Definitionsbereich von $f$ existiert.
\end{definition}


\begin{theorem}
Ist $f$ an der Stelle $x_0$ differenzierbar, so ist $f$ an der Stelle $x_0$ auch stetig.
\\\\
Es folgt: Ist $f$ auf $(a,b), a,b\in \R$ differenzierbar, so ist $f$ auf $(a,b)$ stetig.
\end{theorem}

\begin{theorem}
Sei $(f_n)$ eine Folge in $C^1(\Omega)$ mit $f_n \to^{glm.} f$, $f'_n \to^{glm.} g$ $(n\to \infty)$ und $f,g: \ \Omega \to \R$.
Dann gilt: $f\in C^1(\Omega)$ und $f' = g$.
\end{theorem}

\begin{theorem}[Regeln zum Differenzieren]
	%\textbf{Satz 5.1.2}
    Seien $f,g: \Omega \rightarrow \mathbb{R}$ an der Stelle $x_0$ differenzierbar. Dann ebenfalls $f+g,f\cdot g$ und falls $g(x_0) \neq 0$ auch $\frac{f}{g}$. Zudem gilt:

	\begin{enumerate}
		\item $(f+g)'(x_0)= f'(x_0)+g'(x_0)$
	
		\item $(fg)'(x_0) = f'(x_0)g(x_0)+f(x_0)g'(x_0)$
		
		\item  $(\frac{f}{g})'(x_0)= \frac{f'(x_0)g(x_0)-f(x_0)g'(x_0)}{g^2(x_0)}$
	\end{enumerate}
\end{theorem}

\begin{theorem}[Kettenregel]
	%\textbf{Satz 5.1.3 (Kettenregel)}
    Seien $f$ an der Stelle $x_0$ und $g$ an der Stelle $y_0 = f(x_0)$ differenzierbar. Dann ist deren Verknüpfung $g \circ f$ an der Stelle $x_0$ differenzierbar. Zudem gilt:
	
	\centering{$(g\circ f)'(x_0) = g'(f(x_0))f'(x_0)$}
	
\end{theorem}

\begin{concept}[Kurvendiskussion]
Sei $f: \Omega \rightarrow \mathbb{R}$, eine hinreichend oft differenzierbare Funktion, $x_0 \in \Omega$ und  $P_0 := (x_0,f(x_0))$.
	\begin{enumerate}
		\item Falls $f'= 0 \Rightarrow f$ ist konstant.
		
        \item Falls $f' \geq 0$ (bzw. $>0$) auf $]a,b[ \Rightarrow f$ ist (streng) monoton wachsend. 
	\end{enumerate}
    
    Des weiteren gilt:
\begin{itemize}
\item $x_0$ ist \textbf{Nullstelle} von $f$ \ $\iff$ \ $f(x_0) = 0$
\item $f'(x) > 0$ $\Rightarrow$ $f$ ist \textbf{streng monoton wachsend}
\item $f'(x) < 0$ $\Rightarrow$ $f$ ist \textbf{streng monoton fallend}
\item $f$ hat bei $x_0$ ein \textbf{lokales Maximum}  \ $\Leftrightarrow$ \ $f'(x_0) = 0$ und $f''(x_0) < 0$
\item $f$ hat bei $x_0$ ein \textbf{lokales Minimum}  \ $\Leftrightarrow$ \ $f'(x_0) = 0$ und $f''(x_0) > 0$
\item $P_0$ ist ein \textbf{Wendepunkt}  \ $\Leftrightarrow$ \ $f''(x_0) = 0$ und $f'''(x_0) \neq 0$
\item $P_0$ ist ein \textbf{Sattelpunkt}  \ $\Leftrightarrow$ \ $f'(x_0) = f''(x_0) = 0$ und $f'''(x_0) \neq 0$\\
\end{itemize}
\end{concept}

\begin{corollary}[Genaueres Kriterium für lokale Minima/Maxima]
Sei $f\in C^m(\Omega)$ und $x_0\in \Omega$. Angenommen, die ersten m-1 Ableitungen von $f$ in $x_0$ sind alle 0, dann gilt:
\begin{enumerate}
\item Falls m ungerade und $x_0$ ein lokales Minimum/Maximum, so folgt $f^{(m)}(x_0)=0$.
\item Falls m gerade und $f^{(m)}>0$ ($< 0$), so ist $x_0$ ein lokales Minimum (Maximum) von $f$.
\end{enumerate}
\end{corollary}

\section{Der Mittelwertsatz und Folgerungen}

\begin{theorem}[Mittelwertsatz]
    Seien $-\infty < a < b < +\infty$. Sei $f: [a,b] \rightarrow \mathbb{R}$ stetig und auf $]a,b[$ differenzierbar.
    Dann gilt

$$\exists x_0 \in ]a,b[: \ f'(x_0) = \frac{f(b)-f(a)}{b-a}$$
oder auch
$$\exists x_0 \in ]a,b[: f(b) = f(a) + f'(x_0)(b-a)$$
\end{theorem}

\begin{theorem}[Bernoulli- de l'Hôpital]
	Seien $f,g: \ ]a,b[ \rightarrow \mathbb{R}$ stetig und differenzierbar in $]a,b[$ mit $g'(x)\neq 0, \forall x\in ]a,b[$. Weiter sei $f(a)= 0 = g(a)$, und es existiere der Grenzwert
	
	\centering{$\lim_{x \to a}\frac{f'(x)}{g'(x)}$}
		
	\raggedright{So gilt:}
	
	\centering{$\lim_{x \to a}\frac{f(x)}{g(x)}=\lim_{x \to a}\frac{f'(x)}{g'(x)}$}

In anderen Worten. Falls $\lim\frac{f(x)}{g(x)}$ von der Form ''$\frac{0}{0}$'' oder ''$\frac{\infty}{\infty}$'' ist, dann ist
$\lim\frac{f(x)}{g(x)}=\lim\frac{f'(x)}{g'(x)}$.
\end{theorem}

\begin{theorem}[Umkehrsatz]
Sei f: $]a,b[ \rightarrow \mathbb{R}$ differenzierbar mit f' $>$  0 auf $]a,b[$, und seien
	
	\centering{$-\infty  \leq c = inf_{a<x<b} f(x) < sup_{a<x<b} f(x) = d 		\leq \infty$}

	\raggedright Dann ist f: $]a,b[ \rightarrow ]c,d[$ bijektiv und die 		Umkehrfunktion $f^{-1}:  ]c,d[ \rightarrow \mathbb{R}$ ist differenzierbar mit

	\centering $(f^{-1})'(f(x)) = (f'(x))^{-1},  \forall x \in ]a,b[,$
    
    \raggedright bzw.
    
    \centering $(f^{-1})'(y) = \frac{1}{f'(f^{-1}(y))},  \forall y \in ]c,d[,$
\end{theorem}

\begin{definition}[$C^k$-Räume]
Eine Funktion $f: \Omega \rightarrow \mathbb{R}$ heisst von der Klasse $C^{(m)}(\Omega)$ ($m \in \mathbb{N}_0$), $f \in C^{(m)}$, falls sie auf $\overline{\Omega}$ $m$-mal differezierbar ist und $f^{(m)}$ stetig ist. Man sagt, dass $f$ von der Klasse $C^\infty$ (glatte Funktion) ist, falls $f$ von der Klasse $C^m$ für alle $m \in \mathbb{N_0}$ ist. 
\end{definition}

\begin{definition}[Konvexität]
Eine Funktion $f: ]a,b[ \to \R$ heisst \textbf{konvex}, wenn für alle $x_0, x_1 \in ]a,b[$, $0\leq t \leq 1$:
$$f(tx_1 + (1-t)x_0) \leq tf(x_1) + (1-t)f(x_0)$$
\end{definition}

\begin{theorem}
Ist $f$ zweimal diffbar und $f''$ stetig, so sind äquivalent:
\begin{enumerate}
\item $f$ ist konvex
\item $f'$ ist monoton wachsend
\item $f'' \geq 0$
\item Der Graph von $f$ liegt oberhalb jeder seiner Tangenten.
\end{enumerate}
\end{theorem}

\begin{theorem}[Jensen's Ungleichung]
Sei $f: \ ]a,b[ \to \R$ konvex.
Dann gilt für beliebige Punkte $x_1, .... , x_n \in ]a,b[$ und Zahlen $0\geq t_1,...,t_n \geq 1$ mit $\sum_{i=1}^n t_i = 1$ die Ungleichung:
$$f\Big(\sum_{i=1}^n t_i x_i \Big) \ \leq \ \sum_{i=1}^n t_i f(x_i)$$

\textbf{Beweis:} Induktion über Konvexität
\end{theorem}

\section{Taylor}

\begin{theorem}[Taylor]
Sei $f\in C^{m-1}([a,b])$ auf $]a,b[$ m-mal differenzierbar. Dann folgt:

$$\exists \xi \in ]a,b[:
f(b) = f(a) + f'(a)(b-a) + f''(a)\frac{(b-a)^2}{2!} + ... +$$
$$f^{(m-1)}(a)\frac{(b-a)^{m-1}}{(m-1)!} + f^{(m)}(\xi)\frac{(b-a)^m}{m!}$$
\end{theorem}

\begin{definition}
Wir definieren das \textbf{Taylor-Polynom m-ter Ordnung} als:
$$T_mf(b;a) = f(b) = f(a) + f'(a)(b-a) + f''(a)\frac{(b-a)^2}{2!} + ... +$$
$$f^{(m-1)}(a)\frac{(b-a)^{m-1}}{(m-1)!} + f^{(m)}(a)\frac{(b-a)^m}{m!}$$

Das Taylor-Polynom können wir natürlich auch noch als Summenformel schreiben:
$$T_mf(b;a) = \sum_{k=0}^m f^{(k)}\cdot\frac{(b-a)^k}{k!}$$
\end{definition}

\begin{concept}[Abschätzung des Fehlers]
Die Größe unseres Fehlers wollen wir jetzt abschätzen.
Dafür definieren wir ihn erstmal:

$$r_mf(b;a) := f(b)-T_mf(b;a)$$

%Ob ich hier b oder x schreibe, ist egal. Wir müssen jetzt nur das Intervall $[a,x]$ betrachten.
Der Fehler hier kann natürlich negativ werden, also nehmen wir den Betrag. Bei der Subtraktion der beiden riesigen Formeln bleibt nur jeweils der letzte Term stehen und wir kriegen:
$$\exists \xi \in ]a,b[: |r_mf(b,a)| = |f^{(m)}(\xi)\frac{(b-a)^m}{m!}-f^{(m)}(a)\frac{(b-a)^m}{m!}|$$
$$\leq sup_{\mu \in ]a,[b}|(f^{(m)}(\mu)-f^{(m)}(a))\frac{(b-a)^m}{m!}|$$
$$\leq sup_{\mu \in ]a,[b}|(f^{(m)}(\mu)-f^{(m)}(a))|\frac{(b-a)^m}{m!}$$
weil $b>a$ in unserem Fall,
$$\leq sup_{c \in ]a,b[}|f^{(m+1)}(c)|\frac{(b-a)^{m+1}}{(m+1)!}$$

In der letzten Zeile haben wir den Mittelwertsatz verwendet, was natürlich nur geht, wenn die Funktion mindestens (m+1)-mal diffbar ist.
\end{concept}

\begin{theorem}[Abschätzung der Fehlers]
Ist $I \subset \R$ ein offenes Intervall, $m\in \N$ \& $f\in C^{m+1}(I): I \to \R$.
Dann gilt für $a, b \in I$:
$$r_m(b;a) = \int_a^b\frac{(b-t)^m}{m!} \cdot f^{(m+1)}(t) dt$$
\end{theorem}

\begin{example}[log(3/2) at 1]

$$\log(3/2) \approx log(1) + log'(1)\cdot(3/2 - 1) + log''(1)\cdot\frac{(3/2-1)^2}{2!}$$
$$= 0 + \frac{1}{1}\cdot\frac{1}{2} - \frac{1}{1^2}\cdot\frac{\frac{1}{2^2}}{2}$$
$$= \frac{1}{2} - \frac{1}{8} = \frac{3}{8}$$

Und wie gut ist diese Näherung jetzt? Dafür berechnen wir mit der Formel von oben den Fehler:
$$|r_2log(\frac{3}{2};1)|$$
$$\leq sup_{c \in ]1,3/2[}|log'''(c)|\cdot\frac{(\frac{3}{2}-1)^{3}}{3!}$$
$$= sup_{c \in ]1,3/2[}|\frac{2}{c^3}|\cdot \frac{1}{48} = \frac{2}{1^3}\cdot\frac{1}{48} = \frac{1}{24}$$

Wir haben unseren gesuchten Wert also auf $\frac{1}{24}$ genau getroffen.
$\Rightarrow log(\frac{3}{2})\in [\frac{3}{8}-\frac{1}{24}, \frac{3}{8}+\frac{1}{24}] = [\frac{1}{3}, \frac{5}{12}]$
\end{example}

\section{Newton-Raphson-Verfahren}
Das Newton-Raphson Verfahren dient dazu eine Lösung für die Gleichung $f(x) = 0$ zu finden.\\
Bedingungen:

\begin{enumerate}
	\item $f$ ist stetig
	\item  $f$ ist differenzierbar
	\item  $f'$ ist stetig
\end{enumerate}

Zudem machen wir folgende Annahmen:

\begin{enumerate}
	\item $f: \ ]a,b[\rightarrow \mathbb{R}$ erfüllt $f(a) < 0$ und $f(b) > 0$ (oder umgekehrt)
	
	\item $f'(x) \neq 0$ für $a<x<b$
\end{enumerate}


\underline{\textbf{Input}}

$x_0 \in ]a,b[$\\

\underline{\textbf{Induktion}}


$$x_{n+1} = x_n - \frac{f(x_n)}{f'(x_n)}$$


Hiermit lässt sich folgender Satz zeigen:

\emph{Wenn $x_0$ nah genug an der Lösung der Gleichung $f(x) = 0$ ist, dann ist die Folge $(x_n)_{n\in \mathbb{N}_0}$ definiert und konvergiert gegen die Lösung $y$.}

%%%%%%%%%%%%%%%%%%%%%%%%%%%%%%%%%%%%%%%%%%%%%%%%%%%%%%%%%%
%

%\chapter{Riemann-Integral}

%%%%%%%%%%%%%%%%%%%%%%%%%%%%%%%%%%%%%%%%%%%%%%%%

%% Füllt die Lücken. :)
%% Ihr habt einige Freiheiten dabei, z.B. dürft ihr gerne zusätzliche Dinge hinzufügen etc.
%% Einige Dinge sind vordefiniert (dank eines Freundes von mir), aber wenn ihr andere Formatierungen verwenden wollt, könnt ihr das gerne tun.
%% Bitte fragt mich vorher, wenn ihr größere Änderungen an den config-Dateien vornehmen wollt.


%%%%%%%%%%%%%%%%%%%%%%%%%%%%%%%%%%%%%%%%%%%%%%%%

\section{Stammfunktionen}
\begin{definition}[Stammfunktion einer Funktion]

Sei $f:]a,b[\rightarrow\R$ stetig, also $f\in C^0(]a,b[)$.
Eine Funktion $F\in C^1(]a,b[)$ heißt Stammfunktion von f gdw. $\forall x\in ]a,b[: F'(x)=f(x)$ gilt.
\end{definition}

\begin{remark}
Ist $f$ integrierbar, so muss nicht zwingenderweise eine stetige Stammfunktion existieren.
\end{remark}

\begin{theorem}[Konstante]
Seien $F_1, F_2: ]a,b[\rightarrow \R$ Stammfunktionen von $f\in C^0(]a,b[)$. Dann gilt $F_1-F_2 = c\in\R$.
\end{theorem}

\section{Formale Definition des Riemann=Integrals}

% \begin{definition}[Charakteristische Funktion]
% Sei $\Omega \subset \R$ eine Teilmenge der reellen Zahlen. Dann definieren wir die \textit{charakteristische Funktion} von $\Omega$ als
% $$\chi_\Omega: \Omega\rightarrow\R$$
% $$\chi_\Omega(x) = 
% 	\begin{cases} 
%       1		\hfill & x\in\Omega \\
%       0		\hfill & x\notin\Omega \\
%   \end{cases}$$
% \end{definition}

%%%%%%%%%%%%%%%%%%%%%%%%%%%%%%%%%%%%%%%%%%%%%%%%%%%%%%

\begin{definition}[Partition = Zerlegung]
Sei $I:=[a,b]\subset \R$ ein Intervall. Dann nennt man eine endliche Teilmenge von Punkten in I eine Partition. Wenn man sie ordnet, erhält man $P:=\{x_0,...,x_n\}$, wobei man häufig $x_0=a$ und $x_n=b$ wählt.
\end{definition}

\begin{definition}[Riemann-Summe]
Sei $I:=[a,b]\subset \R$ ein Intervall und $P:=\{x_0,...,x_n\}\subset I$ eine Partition von I. Gegeben sei außerdem eine Menge von Stützpunkten für unsere Partition, also $\Xi := \{\xi_1, ..., \xi_n| x_{i-1}\leq \xi_i\leq x_i\}$. Dann definieren die Riemann-Summe über P als
$$S(f, P, \Xi) = \sum_{i=1}^n\xi_i\cdot(x_i-x_{i-1})$$
\end{definition}

\begin{definition}[Unter- und Obersumme]

Sei f: [a,b] $\rightarrow \R$, beschränkt und P $\subset$ [a,b] eine Zerlegung von [a,b].\\ Wir definieren die Unter- und die Obersumme über spezielle Riemann-Summen.
Die Untersumme ist: 
$$\underline{S}(f,P) = S(f, P, \Xi_-) = \sum_{i=1}^n\inf\limits_{x_{i-1}\leq x \leq x_i} f(x)\cdot(x_i-x_{i-1})$$
Analog ist die Obersumme:
$$\overline{S}(f,P) = S(f, P, \Xi_+) = \sum\limits_{i=1}^n\sup\limits_{x_{i-1}\leq x \leq x_i} f(x)\cdot(x_i-x_{i-1})$$
\end{definition}

\begin{theorem}
Seien $a<b\in\R$, $f: I=[a,b] \to\R$ beschränkt, so gilt:
$$\sup\limits_{P\in\mathcal{P}(I)}\underline{S}(f,P) \leq \inf\limits_{P\in\mathcal{P}(I)}\overline{S}(f,P)$$
\end{theorem}

\begin{definition}[Riemann-Integrierbarkeit]

Eine Funktion f heisst Riemann-integrierbar auf [a,b], falls f beschränkt ist und
$$\sup\limits_{P} \underline{S}(f, P) = \inf\limits_{P} \overline{S}(f, P) =: \int_a^b f(x) dx$$
gilt, wobei P eine Zerlegung von [a,b] ist.
Definieren wir $P_n$ als eine beliebige Partition von [a,b] mit n Teilintervallen und Stützpunkten, erhalten wir außerdem
$$\lim\limits_{n\rightarrow\infty} \underline{S}(f, P_n) = \lim\limits_{n\rightarrow\infty} \overline{S}(f, P_n) =: \int_a^b f(x) dx$$

Für \textbf{Beweise} ist folgende Definition hilfreich:
$$\forall \epsilon > 0 \exists P :|\underline{S}(f,P) - \underline{S}(f,P)| < \epsilon$$
\end{definition}

\begin{definition}[Feinheit]
Wir bezeichnen diese \textbf{Feinheit} einer Zerlegung $P$ mit:
$$\mu(P) := max\{x_i -x_{i-1} |x_i \in P,\; 1 \leq i \leq |P|\}$$
\end{definition}

\begin{theorem}
Sei $f:I=[a,b]\to\R$ beschränkt und $A\in\R$, so sind äquivalent:
\begin{itemize}
\item $f$ ist Riemann-integrierbar und $A = \int_a^bf(x)dx$.
\item $\forall \epsilon > 0 \; \exists P \in \mathcal{P}(I):\quad A-\epsilon < \underline{S}(f,P)\leq \overline{S}(f,P) < A+\epsilon$ $$\Leftrightarrow$$ $\overline{S}(f,P) - \underline{S}(f,P) < \epsilon$
\end{itemize}
\end{theorem}

\begin{theorem}
Sowohl \textbf{stetige} als auch \textbf{monotone} Funktionen $f:[a,b]\rightarrow\R$ ($[a,b]$ kompakt) sind Riemann-integrierbar.
\end{theorem}

%%%%%%%%%%%%%%%%%%%%%%%%%%%%%%%%%%%%%%%%%%%%%%%%%%%%%%

\begin{theorem}[Eigenschaften des Integral]

Seien $f,g: [a,b]\rightarrow \mathbb{R}$ Riemann-integrierbar und sei $\alpha,p \in \mathbb{R}$, $p \geq 1$, zudem $a<c<b$. Dann gilt:
		
	\begin{enumerate}
		\item $f+g$, $fg$, $\alpha f$, $|f|^p$, $\max\{f,g\}$ und $\min\{f,g\}$  sind über $[a,b]$ integrierbar.
	\end{enumerate}
	
	\textbf{Linearität}
	\begin{enumerate}[resume]
		\item $\int_{a}^{b}(\alpha f(x)) \ dx = \alpha \int_{a}^{b}f(x) \ dx $
		
		\item $\int_{a}^{b} (f(x)+g(x)) \ dx = \int_{a}^{b}f(x) \ dx + \int_{a}^{b}g(x) \ dx$
	\end{enumerate}	
	
	\textbf{Monotonie}
	\begin{enumerate}[resume]
		\item $\int_{a}^{b}f(x) \ dx \leq \int_{a}^{b}g(x) \ dx $, falls $f \leq g$
	\end{enumerate}
	
	\textbf{Dreiecksungleichung}
	\begin{enumerate}[resume]
		\item $\big|\int_{a}^{b} f(x) dx\big| \leq \int_{a}^{b}|f(x)| dx$
	\end{enumerate}

	\textbf{Gebiets-Additivität}
	\begin{enumerate}[resume]
		\item $\int_{a}^{b}f(x) dx = \int_{a}^{c} f(x) dx + \int_{a}^{b}f(x) dx$
        \item $b<a: \quad \int_{a}^{b}f(x) dx = -\int_{b}^{a}f(x) dx$
	\end{enumerate}
\end{theorem}

\begin{theorem}[Riemannintegrierbarkeit glm. konv. Funktionsfolgen]
Sei $f_n: I=[a,b] \to \R$ eine Folge Riemannintegrierbarer Funktionen die glm. gegen $f: I\to\R$ konvergieren, dann ist $f$ R-intbar mit $\int_a^b f(x) dx =\lim\limits_{n\to \infty} \int_a^b f_n(x) dx$
\end{theorem}

\begin{theorem}
Sei $f: I \to \R$ beschränkt, d.h. $\exists M\in\R\:\forall x\int I:\; -M \leq f(x) \leq M$, dann gilt für $P,Q\in\mathcal{P}(I)$:
\begin{itemize}
\item $\underline{S}(f,P)\geq \underline{S}(f,Q) - 4M\mu(P)N(Q)$
\item $\overline{S}(f,P)\leq \overline{S}(f,Q) + 4M\mu(P)N(Q)$
\end{itemize}
\end{theorem}

\begin{theorem}[Fundamentalsatz der Analysis]

Seien $a<b \in \mathbb{R}$ und sei $f: [a,b]\rightarrow \mathbb{R}$ eine stetige Funktion. Definieren wir $F: [a,b]\rightarrow \mathbb{R}$ durch

\centering{$F(x) := \int_{a}^{x}f(t) dt$ \tab für $a\leq x\leq b$.}
	
	\raggedright{\textit{Dann ist $F$ stetig differenzierbar und es gilt}}
	
	\centering{ $F'(x) = f(x),\ \forall x \in [a,b]$}\\
	
	
	\raggedright{\textit{Als Korollar folgt daraus:}}
	
	$$\int_{a}^{b} f(t)dt = F(b)- F(a)$$
\end{theorem}

\begin{theorem}[Partielle Integration]

Seien $a < b$ reelle Zahlen 
und seien f, g: [a, b] $\rightarrow \R$ stetig differenzierbar. Dann gilt
$$\int_a^b f(x)g'(x)dx = f(b)g(b) - f(a)g(a) - \int_a^b f'(x)g(x)dx.$$
\end{theorem}

\begin{theorem}[Substitution]

Seien $a < b$ reelle Zahlen, sei $\phi : [a, b] \rightarrow \R$ stetig differenzierbar, sei I $\subset \R$ ein Intervall so dass $\phi([a, b]) \subset I$, und sei f: $I \rightarrow \R$ eine stetige Funktion. Dann gilt
$$\int_{\phi(a)}^{\phi(b)} f(x)dx = \int_a^b f(\phi(t))\phi'(t)dt$$
\end{theorem}

\begin{concept}[Substitution Hints]
	Für $\sin^n$, $\cos^n$ mit n ungerade: $\tan{(t/2)}$.\\
    Für $\sin^n$, $\cos^n$ mit n gerade: $\tan{(t)}$.\\
\end{concept}

%%%%%%%%%%%%%%%%%%%%%%%%%%%%%%%%%%%%%%%%%%%%%%%%%%%%%%

\begin{theorem}[Partialbruchzerlegung]
Für Bruch mit einem Nenner von teilerfremden Polynomen gibt es eine Zerlegung der folgenden Form:
$$\frac{f(x)}{g_1(x) \cdot...\cdot g_r(x)} = \frac{f_1(x)}{g_1(x)}+ ... + \frac{f_r(x)}{g_r(x)}$$
\\\\
Für jede Nullstelle erstellen wir einen Partialbruch mit folgendem Ansatz:
\begin{enumerate}
\item Einfache, reelle Nullstelle:\\
		$x_1 \rightarrow \frac{A}{x-x_1}$
\item r-fache, reelle Nullstelle:\\
		$x_1 \rightarrow \frac{A_1}{x-x_1} + \frac{A_2}{(x-x_1)^2} + ... + \frac{A_r}{(x-x_1)^r}$
\item Einfache, komplexe Nullstelle:\\
		$x^2 + px + q \rightarrow \frac{Ax + B}{x^2+px+q}$
\item r-fache, komplexe Nullstelle:\\
		$x^2 + px + q \rightarrow \frac{A_1x + B_1}{x^2+px+q} + \frac{A_2x + B_2}{(x^2+px+q)^2} + ... + \frac{A_rx + B_r}{(x^2+px+q)^r}$
\end{enumerate}
\end{theorem}

\begin{corollary}
Sei $I\subset \mathbb{R}$ ein Intervall, sei $f:I\rightarrow \mathbb{R}$ eine stetige Funktion, und seien $a,b \in I$ mit $a<b$ und $c\in \mathbb{R}\setminus\{0\}$.

Dann gilt folgendes:

Sind $a+c,b+c \in I$, gilt
	\begin{enumerate}
		\item $\int_{a}^{b}f(t+c)dt = \int_{a+c}^{b+c}f(x)dx$
	\end{enumerate}
	
	\textit{Sind $ca,cb \in I$, gilt}
	\begin{enumerate}[resume]
		\item $\int_{a}^{b}f(ct)dt = \frac{1}{c}\int_{ca}^{cb}f(x)dx$
	\end{enumerate}	

	\textit{Ist $f$ stetig differenzierbar und $f(t) \neq 0,\ \forall t \in [a,b]$, gilt}
	\begin{enumerate}[resume]
	\item $\int_{a}^{b}\frac{f'(t)}{f(t)}dt = \log(|f(b)|)- \log(|f(a)|)$
	\end{enumerate}	
\end{corollary}


\begin{theorem}[Mitelwertsatz der Integralrechnung]
Sei $f$ auf $[a,b]$ stetig, so $\exists \xi \in [a,b]$, sodass gilt
	
	$$\int_{a}^{b}f(x)dc = (b-a) \cdot f(\xi)$$
\end{theorem}


\begin{theorem}
Sei $f:[a,b]\rightarrow \mathbb{R}^2 \in C^1$ und $g: [c,d]\rightarrow [a,b]$ streng wachsend, bijektiv und differenzierbar. Dann haben die parametrischen Kurven $f$ und $f\circ g$ die selben Lösungen.
\end{theorem}


\section{Flächeninhalt, Bogenlänge, Volumen}


Gegeben Funktion $f(x)$ stetig, differenzierbar.
\\\noindent$A  = \int_{a}^{b}f(x)dx,\ f(x) \geq 0$\\
$s = \int_{a}^{b}\sqrt{1+(f'(x))^2}dx = \int_{a}^{b}||f'(t)||dt$\\
$V_x = \pi \int_{a}^{b}(f(x))^2dx$ (Rotationsvolumen um X-Achse)\\
$V_y = \pi \bigg|\int_{a}^{b}x^2f'(x)dx\bigg| = \pi\bigg|\int_{f(a)}^{f(b)}x^2 dy\bigg|, \ f $ monoton\ (Rotationsvolumen um Y-Achse)\\
$M = 2\pi \int_{a}^{b}f(x)\sqrt{1+(f'(x))^2}dx\ (Mantelflaeche)$

\section{Uneigentlicher Integral}

\begin{concept}[Uneigentliches Integral]
	Oft ist es nützlich eine stetige Funktion auf einem offenen Intervall über das gesamte Intervall zu integrieren, selbst wenn das Intervall, bzw. die Funktion, unbeschränkt ist. In diesem Fall sprechen wir von einem \textbf{uneigentlichen (Riemannschen) Integral}.
\end{concept}

\begin{definition}[Uneigentliche Integral]
	Seien $a\in \R \cup \{-\infty\}$ und $b\in \R \cup \{\infty\}$ mit $a < b$ gegeben und sei $I := (a,b)$.
    \\\\
    Eine Funktion $f: I \to \R$ ist \textbf{lokal Riemann integrierbar}, wenn sie auf jedes kompakte Intervall $K \subset I$ Riemann integrierbar ist.
    \\\\
    Eine solche Funktion $f: I \to \R$ ist \textbf{uneigentlich Riemann integrierbar} wenn eine Zahle $A \in \R$ existiert, so dass:
    $$\forall \epsilon > 0: \ \lim_{\alpha \to a} \lim_{\beta \to b} |A - \int_\alpha^\beta f(x) dx| < \epsilon$$
    
    Existiert eine solches $A$,  so ist sie eindeutig bestimmt und wird das \textbf{uneigentliche Integral} genannt:
    $$\int_a^b f(x) dx := A = \lim_{\substack{\alpha \searrow a \\ \beta \nearrow b}} \int_\alpha^\beta f(x) dx$$
\end{definition}

\begin{theorem}[Konvergenz uneigentlicher Integrale]
  \textbf{Integrationsbereich unendlich}
  \begin{enumerate}
      \item \textbf{Direktes Berechnen mittels Definition}
      \item \textbf{Vergleichskriterium} Sei $f,g$ auf $[a,\infty)$ stetig mit $0 \leq f(x) \leq g(x)$, dann: Ist $\int_a^{\infty}g(x)dx$ konvergent, so ist auch $\int_a^{\infty}f(x)dx$ konvergent.
      \item \textbf{Absolute Konvergenz} Ist $\int_a^{\infty}|f(x)|dx$ konvergent, so ist auch $\int_a^{\infty}f(x)dx$ konvergent.
      \item \textbf{Grenzwert-Tests}
      \begin{enumerate}
      	\item $f(x), g(x)$ auf $[a,\infty)$ stetig und $\lim\limits_{x\rightarrow\infty}\frac{f(x)}{g(x)} = A \neq \infty$, dann $\int_a^{\infty}|g(x)|dx$ kovergiert $\Rightarrow$ $\int_a^{\infty}|f(x)|dx$ konvergiert.
        \item $f(x)$ auf $[a,\infty)$ stetig und $\lim\limits_{x\rightarrow\infty}x^p f(x) = A \neq \infty$ für ein $p > 1$, so konvergiert $\int_a^{\infty}|f(x)|dx$ und somit $\int_a^{\infty}f(x)dx$ konvergiert.
        \item $f(x)$ auf $[a,\infty)$ stetig und $\lim\limits_{x\rightarrow\infty}x f(x) = A \neq 0,\infty$, so divergiert $\int_a^{\infty}f(x)dx$.
		\end{enumerate}
        \item \textbf{Leibniz Kriterium} Sei $f(x)$ auf $[a,\infty)$ stetig. Ist $f(x)$ monoton fallen und gilt $\lim\limits_{x\rightarrow+\infty}=0$, so konvergieren die uneigentlichen Integrale $\int_{a}^{\infty}f(x) sin(x) dx$ \& $\int_{a}^{\infty}f(x) cos(x) dx$.
  \end{enumerate}
  \textbf{Integration gegen Polstelle}\\
  Die Kriterien 1-4 lassen sich genau gleich anwenden, 5 geht nicht mehr.
\end{theorem}

\begin{theorem}[Eigenschaften des uneigentlichen Integrals]
	\begin{enumerate}
    \item \textbf{(Kompatibilität)} Falls $I$ kompakt ist, ist $f: I \to \C $ u.R.i gdw. $f$ R.-integrierbar auf $I$ ist. Die Integrale sind gleich.
    \item Falls $a_0<x_0<b_0$, $x_0 \in I$ und sowohl $\int_{x_0}^{b_0} f(t) dt$ als auch $\int_{a_0}^{x_0} f(t) dt$ existieren, dann existiert $\int_{a_0}^{b_0} f(t) dt$ und ist gleich $\int_{a_0}^{b_0} f(t) dt = \int_{a_0}^{x_0} f(t) dt + \int_{x_0}^{b_0} f(t) dt$
    \item \textbf{(Linearität)} Falls $f_1 , \ f_2$ auf $I$ i.R.I sind und $u, \ v \ \in \C$, dann ist $uf_1+vf_2$ u.R.i. auf I mit $\int_I (uf_1(t) + vf_2(t)) dt = u\int_I f_1(t) dt +  v\int_I f_2(t) dt$
    \item Sei $f: [a,b] \to \R$ stetig, diffbar auf $]a,b[$ mit Ableitung $f' = g: ]a,b[ \to \R$, dann ist $g$ auf $]a,b[$ u.R.i. mit $\int_a^b g(t) dt = f(b) - f(a)$.
    \end{enumerate}
\end{theorem}

\begin{definition}[Gamma-Funktion]
	Für jede reelle Zahl $x>0$ ist die Funktion $(a,\infty) \to \R, t \mapsto t^{x-1}e^{-t}$ u.R.i..
    Das resultierende Integral wird \textbf{Gamma-Funktion}  $\Gamma : (0,\infty) \to (0,\infty)$
    genannt:
    $$\Gamma(x) := \int_0^\infty t^{x-1}e^{-t} dt$$
\end{definition}

\begin{theorem}[Eigenschaften der Gamma-Funktion] $ $\\
	\begin{enumerate}
	\item $\Gamma(1) = 1$
    \item $\forall x > 0 : \ \Gamma(x+1) = x\Gamma(x)$
    \item $\Gamma(x)  = \lim_{n\to \infty} \frac{n!n^x}{x(x+1)\cdot ... \cdot(x+n)}$
    \item Für alle natürlichen Zahlen gilt:  $\Gamma(n + 1) = n!$
    \item $\Gamma(x) \cdot \Gamma(1-x) = \frac{\pi}{sin(\pi x)}$
	\end{enumerate}
\end{theorem}
\end{document}


%\appendix


%\backmatter

%\bibliographystyle{plain}
%\bibliography{bib/refs}

\end{document}
 
